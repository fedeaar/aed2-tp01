\begin{tad}{\tadNombre{Mapa}}
\tadIgualdadObservacional{m}{m'}{Mapa}{
    horizontales($m$) $\igobs$ horizontales($m'$) $\yluego$ \\
    verticales($m$) $\igobs$ verticales($m'$) 
}

\tadGeneros{Mapa}
\tadExporta{{\completar}}
\tadUsa{{\completar}}

\tadAlinearFunciones{horizontales}{Mapa}

\tadObservadores
\tadOperacion{horizontales}{Mapa}{conj(Nat)}{}
\tadOperacion{verticales}{Mapa}{conj(Nat)}{}

\tadAlinearFunciones{crear}{conj(Nat),conj(Nat)}{Mapa}{}

\tadGeneradores
\tadOperacion{crear}{conj(Nat),conj(Nat)}{Mapa}{}

\tadAxiomas[\paratodo{conj(Nat)}{hs,vs}]
\tadAlinearAxiomas{horizontales(crear(hs, vs))}

\tadAxioma{horizontales(crear(hs, vs))}{hs}
\tadAxioma{verticales(crear(hs, vs))}{vs}
\end{tad}

\begin{tad}{\tadNombre{SimCity}}
\tadIgualdadObservacional{s}{s'}{SimCity}{
    mapa($s$) $\igobs$ mapa($s'$) $\yluego$ \\
    casas($s$) $\igobs$ casas($s'$) $\wedge$ \\
    comercios($s$) $\igobs$ comercios($s'$) $\wedge$ \\
    popularidad($s$) $\igobs$ popularidad($s'$)  
}

\tadGeneros{SimCity}
\tadExporta{{\completar}}
\tadUsa{{\completar}}

\tadAlinearFunciones{popularidad}{SimCity}

\tadObservadores
\tadOperacion{mapa}{SimCity}{Mapa}{}
\tadOperacion{casas}{SimCity}{dicc(Pos, Nivel)}{}
\tadOperacion{comercios}{SimCity}{dicc(Pos, Nivel)}{}
\tadOperacion{popularidad}{SimCity}{Nat}{}

\tadAlinearFunciones{avanzarTurno}{SimCity/s,dicc(Pos,Construccion)/cs}

\tadGeneradores
\tadOperacion{iniciar}{Mapa}{SimCity}{}
\tadOperacion{avanzarTurno}{SimCity/s,dicc(Pos,Construccion)/cs}{SimCity}{
    ($\forall p$ : Pos)
    (def?($p, cs$) $\impluego$ 
        ($\lnot\ p \in$ claves(construcc($s$)) $\wedge$ \\
         $\lnot \pi_{0}(p) \in$ horizontales(mapa($s$)) $\wedge$ $\lnot \pi_{1}(p) \in$ verticales(mapa($s$)) $\wedge$ \\
         (obtener($p, cs$) $\igobs$ 1 $\vee$ obtener($p, cs$) $\igobs$ 2)))
}
%posiciones no estan ocupadas y no son ríos
\tadOperacion{unir}{SimCity/a,SimCity/b}{SimCity}{
    ($\forall p$ : Pos)
    (def?($p$, construcc($a$)) $\impluego$ \\
    ($\lnot \pi_{0}(p) \in$ horizontales(mapa($b$)) $\wedge$ $\lnot \pi_{1}(p) \in$ verticales(mapa($b$)) $\wedge$ \\
    ($p \in$ masNivel($a$) $\implies \lnot p \in$ construcc($b$)))) $\wedge$ \\
    ($\forall p$ : Pos)
    (def?($p$, construcc($b$)) $\impluego$ \\
    ($\lnot \pi_{0}(p) \in$ horizontales(mapa($a$)) $\wedge$ $\lnot \pi_{1}(p) \in$ verticales(mapa($a$)) $\wedge$ \\
    ($p \in$ masNivel($b$) $\implies \lnot p \in$ construcc($a$))))
}
%ríos no elimnan construcciones y no se pisan construcciones de nivel máximo
\newpage
\tadOtrasOperaciones
\tadOperacion{turnos}{SimCity}{Nat}{}
%Junta los diccionarios casas y comercios
\tadOperacion{construcc}{SimCity}{dicc(Pos, Nivel)}{}
\tadOperacion{unirDicc}{dicc(Pos, Nivel),dicc(Pos, Nivel)}{dicc(Pos, Nivel)}{}
%Calcula las construcciones de mayor nivel
\tadOperacion{masNivel}{SimCity}{conj(Pos)}{}
\tadOperacion{masNivelAux}{dicc(Pos, Nivel),Nat}{conj(Pos)}{}
%Calcula el nivel mas alto de entre todas las construcciones
\tadOperacion{nivelMaximo}{dicc(Pos, Nivel)}{Nat}{}
%Junta dos diccionarios de casas
\tadOperacion{agCasas}{dicc(Pos, Nivel),dicc(Pos, Construccion)}{dicc(Pos,Nivel)}{}
%Junta dos diccionarios de comercios
\tadOperacion{agComercios}{dicc(Pos, Nivel),dicc(Pos, Construccion)}{dicc(Pos,Nivel)}{}
%Calcula el nivel de un comercio al ser agregado al SimCity
\tadOperacion{nivelCom}{Pos,SimCity}{Nat}{}
%Devuelve el diccionario de conostrucciones a distancia manhattan <= 3 
\tadOperacion{conjManhatt}{Pos,dicc(Pos, Nivel)}{dicc(Pos, Nivel)}{}
%Calcula la distancia Manhattan de dos posiciones
\tadOperacion{distManhatt}{Pos,Pos}{Nat}{}
%Devuelve el segundo diccionario sin las claves que tambien aparecian en el primero
\tadOperacion{sacarRepes}{dicc(Pos, Construccion),dicc(Pos, Construccion)}{dicc(Pos, Construccion)}{}

\tadAxiomas[\paratodo{simcity}{s, s'}, \paratodo{dicc(Pos, Construccion)}{cs, cs'}]
\tadAlinearAxiomas{popularidad(avanzarTurno($s, cs$))}

\tadAxioma{mapa(iniciar(m))}{m}
\tadAxioma{mapa(avanzarTurno($s, cs$))}{mapa($s$)}
\tadAxioma{mapa(unir($s, s'$))}{
    crear(horizontales($s$) $\cup$ horizontales($s'$), \\
    verticales($s$) $\cup$ verticales($s'$))
}

\tadAxioma{casas(iniciar(m))}{vacio}
\tadAxioma{casas(avanzarTurno($s, cs$))}{agCasas(casas($s$), $cs$)}
\tadAxioma{casas(unir($s, s'$))}{
    agCasas(casas($s$), sacarRepes(construcc($s$), construcc($s'$)))
}

%Como no se indica qué significa Construccion, asumo que es un nat
%donde 1 = casa, 2 = comercio
%Tecnicamente, agCasas serviria para unir un dicc de casas con uno de
%casas y comercios. Terminaria quedando un dicc con las casas viejas + 
%casas nuevas (no se definen los comercios de haber)
\tadAxioma{agCasas($cs, cs'$)}{
    $\LIF$ $vacio?$(claves($cs'$)) $\LTHEN$ $cs$ $\LELSE$ \\
        $\LIF$ obtener(dameUno(claves($cs'$)), $cs'$) $\igobs$ 1 $\LTHEN$ \\
            agCasas(definir(dameUno(claves($cs'$)), 1, $cs$), \\
                    borrar(dameUno(claves($cs'$)), $cs'$)) \\
        $\LELSE$ \\
            agCasas($cs$, borrar(dameUno(claves($cs'$)), $cs'$)) \\
        $\LFI$
    $\LFI$
}

\tadAxioma{comercios(iniciar(m))}{vacio}
\tadAxioma{comercios(avanzarTurno($s, cs$))}{agComercios(comercios($s$), $cs$)}
\tadAxioma{comercios(unir($s, s'$))}{unirConstrucc($s$, casas($s$, casas($s'$)))}
%Similar a agCasas, pero aparte chequea distancia Manhattan para el nivel.
%Terminaria quedando un dicc con los comercios viejos + 
%comercios nuevos (no se definen las casas de haber)
\tadAxioma{agComercios($cs, cs'$)}{
    $\LIF$ $vacio?$(claves($cs'$)) $\LTHEN$ $cs$ $\LELSE$ \\
        $\LIF$ obtener(dameUno(claves($cs'$)), $cs'$) $\igobs$ 2 $\LTHEN$ \\
            agComercios(definir(dameUno(claves($cs'$)), \\
            nivelCom(dameUno(claves($cs'$)), $s$), $cs$), \\
            borrar(dameUno(claves($cs'$)), $cs'$)) \\
        $\LELSE$ \\
            agComercios($cs$, borrar(dameUno(claves($cs'$)), $cs'$)) \\
        $\LFI$
    $\LFI$
}
\tadAxioma{nivelCom($p, s$)}{
    $\LIF$ $\lnot vacio?$(manhattan($p$, casas($s$))) $\LTHEN$\\
        nivelMaximo(manhattan($p$), casas($s$)) \\
    $\LELSE$
        1
    $\LFI$
}
\tadAxioma{conjManhatt($p, cs$)}{
    $\LIF vacio?$($cs$) $\LTHEN$ 
        $\emptyset$ 
    $\LELSE$ \\
        $\LIF$ distManhatt($p$, dameUno(claves($cs$))) $\leq$ 3 $\LTHEN$ \\
            definir(dameUno(claves($cs$)), 
                    obtener(dameUno(claves($cs$)), $cs$), \\
                    conjManhatt($p$, borrar(dameUno(claves($cs$)), $cs$))) \\
        $\LELSE$ \\
            conjManhatt($p$, borrar(dameUno(claves($cs$)), $cs$)) \\
    $\LFI$
}
%la distancia manhattan entre dos puntos p y q es
%d(p,q) = |p0 - q0| + |p1 - q1|
\tadAxioma{distManhatt($p, q$)}{
    $\LIF\ \pi_{0}(p) < \pi_{0}(q)\ \LTHEN\ q - p\ \LELSE\ p - q\ \LFI$ \\
    + \\
    $\LIF\ \pi_{1}(p) < \pi_{1}(q)\ \LTHEN\ q - p\ \LELSE\ p - q\ \LFI$
}

\tadAxioma{popularidad(iniciar(m))}{0}
\tadAxioma{popularidad(avanzarTurno($s, cs$))}{popularidad($s$)}
\tadAxioma{popularidad(unir($s, s'$))}{popularidad($s$) + 1}

\tadAxioma{turnos(iniciar(m))}{0}
\tadAxioma{turnos(avanzarTurno($s, cs$))}{turnos($s$) + 1}
\tadAxioma{turnos(unir($s, s'$))}{
    $\LIF$ turnos($s$) <$\ $turnos($s'$) $\LTHEN$ turnos($s'$) $\LELSE$ turnos($s$) $\LFI$
}

\tadAxioma{construcc($s$)}{unirDicc(casas($s$), comercios($s$))}
\tadAxioma{unirDicc($cs, cs'$)}{
    $\LIF$ $vacio?$(claves($cs'$)) $\LTHEN$ $cs$ $\LELSE$ \\
    definir(dameUno(claves($cs'$)), \\
    obtener(dameUno(claves($cs'$)), $cs'$), \\
    unirDicc($cs$, borrar(dameUno(claves($cs'$)), $cs'$))) $\LFI$
}

%Tal vez pueda hacerse sin nivelMaximo
\tadAxioma{masNivel($s$)}{masNivelAux(construcc($s$), nivelMaximo(construcc($s$)))}
\tadAxioma{masNivelAux($cs, n$)}{
    $\LIF$ $vacio?$($cs$) $\LTHEN$ $\emptyset$ $\LELSE$ \\
        $\LIF$ obtener(dameUno(claves($cs$)), $cs$) $\igobs$ n $\LTHEN$ \\
            ag(dameUno(claves($cs$)), \\masNivelAux(borrar(dameUno(claves($cs$)), $cs$), n)) \\
        $\LELSE$ \\
            masNivelAux(borrar(dameUno(claves($cs$)), $cs$), n) \\
        $\LFI\ \LFI$   
}
\tadAxioma{nivelMaximo($cs$)}{
    $\LIF$ $vacio?$($cs$) $\LTHEN$ 0 $\LELSE$ \\
    $max$(obtener(dameUno(claves($cs$)), $cs$), \\nivelMaximo(borrar(dameUno(claves($cs$)), $cs$)))
}

%Quiero eliminar las claves que aparecen en ambos dicc en cs'
\tadAxioma{sacarRepes($cs, cs'$)}{
    $\LIF$ $vacio?$(claves($cs$)) $\LTHEN$ $cs'$ $\LELSE$ \\
        $\LIF$ def?(dameUno(claves($cs$)), $cs'$) $\LTHEN$ \\ %Hay repetido
            sacarRepes(borrar(dameUno(claves($cs$)), $cs$), \\
                       borrar(dameUno(claves($cs$)), $cs'$)) \\
        $\LELSE$ \\
            sacarRepes(borrar(dameUno(claves($cs$)), $cs$),  $cs'$) \\
        $\LFI$
    $\LFI$
}
\end{tad}
