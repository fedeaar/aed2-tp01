\begin{tad}{\tadNombre{SimCity}}
    \tadIgualdadObservacional{s}{s'}{SimCity}{
        mapa($s$) $\igobs$ mapa($s'$) $\yluego$ \\
        casas($s$) $\igobs$ casas($s'$) $\wedge$ \\
        comercios($s$) $\igobs$ comercios($s'$) $\wedge$ \\
        popularidad($s$) $\igobs$ popularidad($s'$)  
    }
    
    \tadGeneros{SimCity}
    \tadExporta{SimCity, observadores, generadores, turnos}
    \tadUsa{Mapa, Nat, Pos, Construccion, dicc($\alpha, \beta$), Nivel}
    
    \tadAlinearFunciones{popularidad}{SimCity}
    
    \tadObservadores
    \tadOperacion{mapa}{SimCity}{Mapa}{}
    \tadOperacion{casas}{SimCity}{dicc(Pos, Nivel)}{}
    \tadOperacion{comercios}{SimCity}{dicc(Pos, Nivel)}{}
    \tadOperacion{popularidad}{SimCity}{Nat}{}
    
    \tadAlinearFunciones{avanzarTurno}{SimCity/s,dicc(Pos,Construccion)/cs}
    
    \tadGeneradores
    \tadOperacion{iniciar}{Mapa}{SimCity}{}
    \tadOperacion{avanzarTurno}{SimCity/s,dicc(Pos,Construccion)/cs}{SimCity}{*$avanzarTurnoValido(s,\ cs)$}
    %posiciones no estan ocupadas y no son ríos
    \tadOperacion{unir}{SimCity/a,SimCity/b}{SimCity}{*$unirValido(a,\ b)$}
    %ríos no elimnan construcciones y no se pisan construcciones de nivel máximo
    %\newpage
    \tadOtrasOperaciones
    \tadOperacion{turnos}{SimCity}{Nat}{}
    %Junta los diccionarios casas y comercios
    \tadOperacion{construcc}{SimCity}{dicc(Pos, Nivel)}{}
    \tadOperacion{$\bullet\ \cup_{dicc}\ \bullet$}{dicc($\alpha$, $\beta$),dicc($\alpha$, $\beta$)}{dicc($\alpha$, $\beta$)}{}
    %Calcula las construcciones de mayor nivel
        %\tadOperacion{masNivel}{SimCity}{conj(Pos)}{}
        %\tadOperacion{masNivelAux}{dicc(Pos, Nivel),Nat}{conj(Pos)}{}
    %Calcula el nivel mas alto de entre todas las construcciones
        %\tadOperacion{nivelMaximo}{dicc(Pos, Nivel)}{Nat}{}
    %Junta dos diccionarios de casas
    \tadOperacion{agCasas}{dicc(Pos, Nivel),dicc(Pos, Construccion)}{dicc(Pos,Nivel)}{}
    %Junta dos diccionarios de comercios
    \tadOperacion{agComercios}{SimCity,dicc(Pos, Nivel),dicc(Pos, Construccion)}{dicc(Pos,Nivel)}{}
    %Calcula el nivel de un comercio al ser agregado al SimCity
    \tadOperacion{nivelCom}{Pos,dicc(Pos, Nivel)}{Nat}{}
    %Devuelve el diccionario de conostrucciones a distancia manhattan <= 3 
    %\tadOperacion{conjManhatt}{Pos,dicc(Pos, Nivel)}{dicc(Pos, Nivel)}{}
    %Calcula la distancia Manhattan de dos posiciones
    \tadOperacion{distManhatt}{Pos,Pos}{Nat}{}
    %Devuelve el segundo diccionario sin las claves que tambien aparecian en el primero
    \tadOperacion{sacarRepes}{dicc(Pos, Construccion),dicc(Pos, Construccion)}{dicc(Pos, Construccion)}{}
    
    \tadAxiomas[\paratodo{simcity}{s, s'}, \paratodo{dicc(Pos, Construccion)}{cs, cs'}, \paratodo{dicc(Pos, Nivel)}{cn, cn'}, 
                \paratodo {dicc($\alpha$, $\beta$)}{d, d'}]
    \tadAlinearAxiomas{mapa(avanzarTurno($s, cs$))}
    
    \tadAxioma{mapa(iniciar(m))}{m}
    \tadAxioma{mapa(avanzarTurno($s, cs$))}{mapa($s$)}
    \tadAxioma{mapa(unir($s, s'$))}{
        crear(horizontales($s$) $\cup$ horizontales($s'$), verticales($s$) $\cup$ verticales($s'$))
    } 

    \tadAlinearAxiomas{casas(avanzarTurno($s, cs$))}
    \tadAxioma{casas(iniciar(m))}{vacio}
    \tadAxioma{casas(avanzarTurno($s, cs$))}{agCasas(casas($s$), $cs$)}
    \tadAxioma{casas(unir($s, s'$))}{
        agCasas(casas($s$), sacarRepes(construcc($s$), construcc($s'$)))
    }
    %Como no se indica qué significa Construccion, asumo que es un string
    %donde "casa" = casa, "comercio" = comercio
    %Tecnicamente, agCasas serviria para unir un dicc de casas con uno de
    %casas y comercios. Terminaria quedando un dicc con las casas viejas + 
    %casas nuevas (no se definen los comercios de haber)
    \tadAxioma{agCasas($cn, cs$)}{
        $\LIF$ $vacio?$(claves($cs$)) $\LTHEN$ \\ 
            \tab $cn$ \\
        $\LELSE$ \\
            \tab $\LIF$ obtener(dameUno(claves($cs$)), $cs$) $\igobs$ $"casa"$ $\LTHEN$ \\
                \tab\tab agCasas(definir(dameUno(claves($cs$)), 1, $cn$), \\
                \tab\tab\tab borrar(dameUno(claves($cs$)), $cs$)) \\
            \tab $\LELSE$ \\
                \tab\tab agCasas($cn$, borrar(dameUno(claves($cs$)), $cs$)) \\
            \tab $\LFI$\\
        $\LFI$
    }

    \pagebreak

    \tadAlinearAxiomas{comercios(avanzarTurno($s, cs$))}
    \tadAxioma{comercios(iniciar(m))}{vacio}
    \tadAxioma{comercios(avanzarTurno($s, cs$))}{agComercios($s$, comercios($s$), $cs$)}
    \tadAxioma{comercios(unir($s, s'$))}{
        agComercios($s$, \\
        \tab\tab comercios($s$), \\
        \tab\tab sacarRepes(construcc($s$), construcc($s'$)))
    }
    %Similar a agCasas, pero aparte chequea distancia Manhattan para el nivel.
    %Terminaria quedando un dicc con los comercios viejos + 
    %comercios nuevos (no se definen las casas de haber)
    \tadAlinearAxiomas{agComercios($s, cn, cs$)}
    \tadAxioma{agComercios($s, cn, cs$)}{
        $\LIF$ $vacio?$(claves($cs$)) $\LTHEN$ \\
        \tab $cn$ \\
        $\LELSE$ \\
            \tab $\LIF$ obtener(dameUno(claves($cs$)), $cs$) $\igobs$ $"comercio"$ $\LTHEN$ \\
                \tab\tab agComercios(definir(dameUno(claves($cs$)), \\
                \tab\tab\tab\tab nivelCom(dameUno(claves($cs$)), casas($s$)), $cn$), \\
                \tab\tab\tab\tab borrar(dameUno(claves($cs$)), $cs$)) \\
            \tab $\LELSE$ \\
                \tab\tab agComercios($cn$, borrar(dameUno(claves($cs$)), $cs$)) \\
            \tab $\LFI$ \\
        $\LFI$
    }
    \tadAxioma{nivelCom($p, cn$)}{
    %    $\LIF$ $\lnot vacio?$(manhattan($p$, casas($s$))) $\LTHEN$\\
    %        nivelMaximo(manhattan($p$), casas($s$)) \\
    %    $\LELSE$
    %        1
    %    $\LFI$
    
    $\LIF\ vacio?$(claves($cn$)) $\LTHEN$ \\
        \tab 1 \\
    $\LELSE$ \\
        \tab $\LIF$ distManhatt($p$, dameUno(claves($cn$))) $\leq$ 3 $\LTHEN$ \\
            \tab\tab max(obtener(dameUno(claves($cn$)), $cn$), \\
            \tab\tab\tab nivelCom($p$, borrar(dameUno(claves($cn$)), $cn$))) \\
        \tab $\LELSE$ \\
            \tab\tab nivelCom($p$, borrar(dameUno(claves($cn$)), $cn$)) \\
        \tab $\LFI$ \\ 
    $\LFI$
}
   % \tadAxioma{conjManhatt($p, cs$)}{
   %     $\LIF vacio?$($cs$) $\LTHEN$ 
   %         $\emptyset$ 
   %     $\LELSE$ \\
   %         $\LIF$ distManhatt($p$, dameUno(claves($cs$))) $\leq$ 3 $\LTHEN$ \\
   %             definir(dameUno(claves($cs$)), 
   %                     obtener(dameUno(claves($cs$)), $cs$), \\
   %                     conjManhatt($p$, borrar(dameUno(claves($cs$)), $cs$))) \\
   %         $\LELSE$ \\
   %             conjManhatt($p$, borrar(dameUno(claves($cs$)), $cs$)) \\
   %     $\LFI$
   % }
    %la distancia manhattan entre dos puntos p y q es
    %d(p,q) = |p0 - q0| + |p1 - q1|
    \tadAxioma{distManhatt($p, q$)}{
        $\LIF\ \pi_{0}(p) < \pi_{0}(q)\ \LTHEN\ q - p\ \LELSE\ p - q\ \LFI$ \\
        + \\
        $\LIF\ \pi_{1}(p) < \pi_{1}(q)\ \LTHEN\ q - p\ \LELSE\ p - q\ \LFI$
    }

    \tadAlinearAxiomas{popularidad(avanzarTurno($s, cs$))}
    \tadAxioma{popularidad(iniciar(m))}{0}
    \tadAxioma{popularidad(avanzarTurno($s, cs$))}{popularidad($s$)}
    \tadAxioma{popularidad(unir($s, s'$))}{popularidad($s$) + 1 + popularidad $s'$}
    
    \tadAlinearAxiomas{turnos(avanzarTurno($s, cs$))}
    \tadAxioma{turnos(iniciar(m))}{0}
    \tadAxioma{turnos(avanzarTurno($s, cs$))}{turnos($s$) + 1}
    \tadAxioma{turnos(unir($s, s'$))}{
        $\LIF$ turnos($s$) <$\ $turnos($s'$) $\LTHEN$ turnos($s'$) $\LELSE$ turnos($s$) $\LFI$
    }

    \tadAxioma{construcc($s$)}{casas($s$) $\cup_{dicc}$ comercios($s$)}
    
    \tadAlinearAxiomas{sacarRepes($cs, cs'$)}

    \tadAxioma{$d$\ $\cup_{dicc}$ $d'$}{
        $\LIF$ $vacio?$(claves($d'$)) $\LTHEN$ \\
            \tab $d$ \\
        $\LELSE$ \\
            \tab definir(dameUno(claves($d'$)), \\
            \tab\tab obtener(dameUno(claves($d'$)), $d'$), \\
            \tab\tab $d$ $\cup_{dicc}$ borrar(dameUno(claves($d'$)), $d'$)) \\
        $\LFI$
    }
    
    %Tal vez pueda hacerse sin nivelMaximo
    %\tadAxioma{masNivel($s$)}{masNivelAux(construcc($s$), nivelMaximo(construcc($s$)))}
    %\tadAxioma{masNivelAux($cs, n$)}{
    %    $\LIF$ $vacio?$($cs$) $\LTHEN$ $\emptyset$ $\LELSE$ \\
    %        $\LIF$ obtener(dameUno(claves($cs$)), $cs$) $\igobs$ n $\LTHEN$ \\
    %            ag(dameUno(claves($cs$)), \\masNivelAux(borrar(dameUno(claves($cs$)), $cs$), n)) \\
    %        $\LELSE$ \\
    %            masNivelAux(borrar(dameUno(claves($cs$)), $cs$), n) \\
    %        $\LFI\ \LFI$   
    %}
    %\tadAxioma{nivelMaximo($cs$)}{
    %    $\LIF$ $vacio?$($cs$) $\LTHEN$ 0 $\LELSE$ \\
    %    $max$(obtener(dameUno(claves($cs$)), $cs$), \\nivelMaximo(borrar(dameUno(claves($cs$)), $cs$)))
    %}
    
    %Quiero eliminar las claves que aparecen en ambos dicc en cs'
    \tadAxioma{sacarRepes($cs, cs'$)}{
        $\LIF$ $vacio?$(claves($cs$)) $\LTHEN$ \\
            \tab $cs'$ \\
        $\LELSE$ \\
            \tab $\LIF$ def?(dameUno(claves($cs$)), $cs'$) $\LTHEN$ \\ %Hay repetido
                \tab\tab sacarRepes(borrar(dameUno(claves($cs$)), $cs$), \\
                \tab\tab\tab\tab borrar(dameUno(claves($cs$)), $cs'$)) \\
            \tab $\LELSE$ \\
                \tab\tab sacarRepes(borrar(dameUno(claves($cs$)), $cs$),  $cs'$) \\
            \tab $\LFI$ \\
        $\LFI$
    }
\end{tad}  

\pagebreak

*donde: $\\$
\tadOperacion{avanzarTurnoValido}{SimCity/s,dicc(Pos,Construccion)/cs}{boolean}{}
\tadAxioma{avanzarTurnoValido($s, cs$)}{$
    \lnot vacio?(claves(cs)) \wedge $\\$ 
    (\forall p : Pos) (def?(p, cs) \impluego $\\$
        $\tab$(\lnot p \in claves(construcc(s))\ \wedge $\\$
        $\tab$\lnot \pi_{0}(p) \in horizontales(mapa($s$)) \wedge \lnot \pi_{1}(p) \in verticales(mapa($s$))\ \wedge $\\$
        $\tab$(obtener(p, cs) \igobs "casa" \vee obtener(p, cs) \igobs "comercio")) $\\$
    )
$}

\tadOperacion{unirValido}{Simcity/a,SimCity/b}{boolean}{}
\tadAxioma{unirValido($a, b$)}{$
    (\forall p : Pos)(def?(p, construcc(a)) \impluego $\\$
        $\tab$ (\lnot \pi_{0}(p) \in horizontales(mapa(b)) \wedge \lnot \pi_{1}(p) \in verticales(mapa(b))\ \wedge $\\$
        %$\tab$ (p \in masNivel(a) \implies \lnot p \in construcc(b))) $\\$
        $\tab$ (\lnot\ (\exists\ otra : Pos)(def?(otra, construcc(a))\ $\yluego$ $\\$
            $\tab\tab$ obtener(otra, construcc(a))\ >\ obtener(p, construcc(a))\ $\impluego \\$ 
                $\tab\tab\tab$ \lnot\ def?(p, construcc(b))))) $\\$
    )\ \wedge $\\$
    (\forall p : Pos)(def?(p, construcc(b)) \impluego $\\$
        $\tab$ (\lnot \pi_{0}(p) \in horizontales(mapa(a)) \wedge \lnot \pi_{1}(p) \in verticales(mapa(a))\ \wedge $\\$
        %$\tab$(p \in masNivel(b) \implies \lnot p \in construcc(a))) $\\$
        $\tab$ (\lnot\ (\exists\ otra : Pos)(def?(otra, construcc(b))\ $\yluego$ $\\$
            $\tab\tab$ obtener(otra, construcc(b))\ >\ obtener(p, construcc(b))\ $\impluego \\$ 
                $\tab\tab\tab$ \lnot\ def?(p, construcc(a)))))
    )
$}

