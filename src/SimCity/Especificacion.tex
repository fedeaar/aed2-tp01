\begin{tad}{\tadNombre{SimCity}}
    \vspace{0.5mm}
    \tadGeneros{SimCity}

    \vspace{2mm}
    \tadExporta{SimCity, observadores, generadores, turnos, $\bullet\ \cup_{dicc}\ \bullet$}

    \vspace{2mm}
    \tadUsa{Mapa, Nat, Pos, Construcción, Nivel, dicc($\alpha\ \times\ \beta$)}
    
    \vspace{2mm}
    \tadIgualdadObservacional{s}{s'}{SimCity}{
        mapa($s$)\ $\igobs$\ mapa($s'$)\ $\yluego$\ \\
        casas($s$)\ $\igobs$\ casas($s'$)\ $\wedge$\ \\
        comercios($s$)\ $\igobs$\ comercios($s'$)\ $\wedge$\ \\
        popularidad($s$)\ $\igobs$\ popularidad($s'$)  
    }
    
    \vspace{2mm}
    \tadAlinearFunciones{popularidad}{SimCity}
    
    \tadObservadores
    \tadOperacion{mapa}{SimCity}{Mapa}{}
    \tadOperacion{casas}{SimCity}{dicc(Pos $\times$ Nivel)}{}
    \tadOperacion{comercios}{SimCity}{dicc(Pos $\times$ Nivel)}{}
    \tadOperacion{popularidad}{SimCity}{Nat}{}
    
    \tadAlinearFunciones{avanzarTurno}{SimCity/s,dicc(Pos,Construccion)/cs}
    
    \vspace{2mm}
    \tadGeneradores
    \tadOperacion{iniciar}{Mapa}{SimCity}{}
    \tadOperacion{avanzarTurno}{SimCity/s,dicc(Pos,Construccion)/cs}{SimCity}{avanzarValido($s$,\ $cs$)$^{1}$}
    %posiciones no estan ocupadas y no son ríos
    \tadOperacion{unir}{SimCity/a,SimCity/b}{SimCity}{unirValido($a$,\ $b$)$^{1}$}
    %ríos no elimnan construcciones y no se pisan construcciones de nivel máximo
    %\newpage

    \vspace{2mm}
    \tadOtrasOperaciones
    \tadAlinearFunciones{construcciones}{dicc($\alpha$, $\beta$),dicc($\alpha$, $\beta$)}
    \tadOperacion{turnos}{SimCity}{Nat}{}
    %Junta los diccionarios casas y comercios
    \tadOperacion{construcciones}{SimCity}{dicc(Pos $\times$ Nivel)}{}
    %Calcula las construcciones de mayor nivel
        %\tadOperacion{masNivel}{SimCity}{conj(Pos)}{}
        %\tadOperacion{masNivelAux}{dicc(Pos, Nivel),Nat}{conj(Pos)}{}
    %Calcula el nivel mas alto de entre todas las construcciones
        %\tadOperacion{nivelMaximo}{dicc(Pos, Nivel)}{Nat}{}
    %Junta dos diccionarios de casas
    \tadAlinearFunciones{construcciones}{SimCity,dicc(Pos, Nivel),dicc(Pos $\times$ Construcción)}
    
    \tadOperacion{agCasas}{dicc(Pos, Nivel),dicc(Pos, Construcción)}{dicc(Pos $\times$ Nivel)}{}
    %Junta dos diccionarios de comercios
    \tadOperacion{agComercios}{dicc(Pos, Nivel),dicc(Pos, Construcción)}{dicc(Pos $\times$ Nivel)}{}
    %Calcula el nivel de un comercio al ser agregado al SimCity
    \tadOperacion{nivelComercio}{Pos,Nat,dicc(Pos, Nivel)}{Nat}{}
    %Calcula la distancia Manhattan de dos posiciones
    \tadOperacion{distManhattan}{Pos,Pos}{Nat}{}
    \tadOperacion{$\bullet\ \cup_{dicc}\ \bullet$}{dicc($\alpha$, $\beta$),dicc($\alpha$, $\beta$)}{dicc($\alpha$ $\times$ $\beta$)}{}

    %\vspace{2mm}
    
    %Devuelve el diccionario de conostrucciones a distancia manhattan <= 3 
    %\tadOperacion{conjManhatt}{Pos,dicc(Pos, Nivel)}{dicc(Pos, Nivel)}{}
     %Devuelve el segundo diccionario sin las claves que tambien aparecian en el primero
    \tadAlinearFunciones{avanzarNivel}{dicc(Pos, Construcción),dicc(Pos, Construcción)}
    \tadOperacion{sacarRepes}{dicc(Pos, Construcción),dicc(Pos, Construcción)}{dicc(Pos $\times$ Construcción)}{}
    \tadOperacion{avanzarNivel}{dicc(Pos, Nivel)}{dicc(Pos, Nivel)}{}
    \tadAlinearFunciones{manhatizar}{dicc(Pos, Construccion),dicc(Pos, Nivel),dicc(Pos, Nivel)}
    \tadOperacion{unirConst}{dicc(Pos, Construccion),dicc(Pos, Nivel),dicc(Pos, Nivel)}{dicc(Pos, Nivel)}{}
    \tadOperacion{manhatizar}{dicc(Pos, Nivel),dicc(Pos, Nivel)}{dicc(Pos, Nivel)}{}
   
    \vspace{2mm}
    %\newpage
    \tadAxiomas[
        \paratodo{simcity}{s,\ s'}, 
        \paratodo{Mapa}{m},
        \paratodo{dicc(Pos $\times$ Construcción)}{cs}
        ]
    \vspace{1mm}
    \tadAlinearAxiomas{mapa(avanzarTurno($s,\ cs$))}
    
    \tadAxioma{mapa(iniciar($m$))}{m}
    \tadAxioma{mapa(avanzarTurno($s,\ cs$))}{mapa($s$)}
    \tadAxioma{mapa(unir($s,\ s'$))}{
        mapa($s$) + mapa($s'$)
    } 

    \vspace{2mm}
    \tadAlinearAxiomas{casas(avanzarTurno($s,\ cs$))}
    \tadAxioma{casas(iniciar($m$))}{vacio}
    \tadAxioma{casas(avanzarTurno($s,\ cs$))}{agCasas(avanzarNivel(casas($s$)),\ $cs$)}
    \tadAxioma{casas(unir($s,\ s'$))}{
        %agCasas(casas($s$), sacarRepes$^{2}$(construcciones($s$), construcciones($s'$)))
        unirConst(construcciones($s$), casas($s$), casas($s'$))
    }

    \vspace{2mm}
    \tadAlinearAxiomas{popularidad(avanzarTurno($s,\ cs$))}
    \tadAxioma{comercios(iniciar($m$))}{vacio}
    \tadAxioma{comercios(avanzarTurno($s,\ cs$))}{
        manhatizar(casas($s$), \\
        \tab agComercios(avanzarNivel(comercios($s$)), $cs$))
    }
    \tadAxioma{comercios(unir($s,\ s'$))}{
       % agComercios($s$, comercios($s$), \\
       % \tab\tab\tab sacarRepes$^{2}$(construcciones($s$), construcciones($s'$)))
        manhatizar(casas(unir($s,\ s'$)), \\
        \tab unirConst(construcciones($s$), comercios($s$), comercios($s'$)))
    }

    \vspace{2mm}
    \tadAlinearAxiomas{popularidad(avanzarTurno($s,\ cs$))}
    \tadAxioma{popularidad(iniciar($m$))}{0}
    \tadAxioma{popularidad(avanzarTurno($s,\ cs$))}{popularidad($s$)}
    \tadAxioma{popularidad(unir($s,\ s'$))}{popularidad($s$) + 1 + popularidad($s'$)}
    
    \vspace{2mm}
    \tadAlinearAxiomas{turnos(avanzarTurno($s,\ cs$))}
    \tadAxioma{turnos(iniciar($m$))}{0}
    \tadAxioma{turnos(avanzarTurno($s,\ cs$))}{turnos($s$) + 1}
    \tadAxioma{turnos(unir($s,\ s'$))}{
        $\LIF$ turnos($s$) <$\ $turnos($s'$) $\LTHEN$ turnos($s'$) $\LELSE$ turnos($s$) $\LFI$
    }

    \vspace{2mm}
    \tadEncabezado{otros ax.}{\paratodo{simcity}{s}, \paratodo{Pos}{p,\ q}, \paratodo{dicc(Pos $\times$ Construcción)}{cs,\ cs'}, 
    \paratodo{dicc(Pos $\times$ Nivel)}{cn,\ cn'}, \\
    \paratodo {dicc($\alpha$ $\times$ $\beta$)}{d,\ d'}
    }
    \vspace{1mm}
    \tadAlinearAxiomas{agComercios($s,\ cn,\ cs$)}

    \tadAxioma{construcciones($s$)}{casas($s$) $\cup_{dicc}$ comercios($s$)}
    
    \tadAxioma{agCasas($cn,\ cs$)}{
        $\LIF$ vacio?(claves($cs$)) $\LTHEN$ \\ 
            \tab $cn$ \\
        $\LELSE$ $\LIF$ obtener(proximo, $cs$) = $"casa"$ $\LTHEN$ \\
            \tab agCasas(definir(proximo, 1, $cn$), borrar(proximo, $cs$)) \\
        $\LELSE$ \\
        \tab agCasas($cn$, borrar(proximo, $cs$)) \\
        $\LFI$
        \\ donde proximo $\equiv$ dameUno(claves($cs$)) 
    }

    \tadAxioma{agComercios($cn,\ cs$)}{
        %$\LIF$ vacio?(claves($cs$)) $\LTHEN$ \\
        %\tab $cn$ \\
        %$\LELSE$ $\LIF$ obtener(proximo, $cs$) = $"comercio"$ $\LTHEN$ \\
        %        \tab agComercios(s, definir(proximo, nivelComercio(proximo, casas($s$)), $cn$), \\
        %        \tab\tab\tab\tab borrar(proximo $cs$)) \\
        %$\LELSE$ \\
        %\tab agComercios(s, $cn$, borrar(proximo, $cs$)) \\
        %$\LFI$ 
        %\\ donde proximo $\equiv$ dameUno(claves($cs$)) 
        $\LIF$ vacio?(claves($cs$)) $\LTHEN$ \\ 
            \tab $cn$ \\
        $\LELSE$ $\LIF$ obtener(proximo, $cs$) = $"comercio"$ $\LTHEN$ \\
            \tab agComercios(definir(proximo, 1, $cn$), borrar(proximo, $cs$)) \\
        $\LELSE$ \\
            \tab agComercios($cn$, borrar(proximo, $cs$)) \\
        $\LFI$
        \\ donde proximo $\equiv$ dameUno(claves($cs$)) 
    }
    \tadAlinearAxiomas{nivelComercio($p,\ n,\ cn$)}
    \tadAxioma{nivelComercio($p,\ n,\ cn$)}{
        $\LIF$\ vacio?(claves($cn$)) $\LTHEN$ \\
            \tab $n$ \\
        $\LELSE$ $\LIF$ distManhattan($p$, proximo) $\leq$ 3 $\wedge$ obtener(proximo, $cn$) > $n$ $\LTHEN$ \\
            \tab nivelComercio($p$, obtener(proximo, $cn$), borrar(proximo, $cn$)) \\
        $\LELSE$ \\
            \tab nivelComercio($p$, $n$, borrar(proximo, $cn$)) \\
        $\LFI$  
        \\ donde proximo $\equiv$ dameUno(claves($cn$)) 
    }

    \tadAlinearAxiomas{agComercios($s,\ cn,\ cs$)}
    %la distancia manhattan entre dos puntos p y q es
    %d(p,q) = |p0 - q0| + |p1 - q1|
    %\vspace{2mm}
    \tadAxioma{distManhattan($p,\ q$)}{
        %$\LIF\ \pi_{0}(p) < \pi_{0}(q)\ \LTHEN\ q - p\ \LELSE\ p - q\ \LFI$ \\
        %+ \\
        %$\LIF\ \pi_{1}(p) < \pi_{1}(q)\ \LTHEN\ q - p\ \LELSE\ p - q\ \LFI$
        $\LIF$\ $p.x$ <\ $q.x$\ $\LTHEN$\ $q.x$ - $p.x$\ $\LELSE$\ $p.x$ - $q.x$\ $\LFI$ $\\$
        + \\
        $\LIF$\ $p.y$ <\ $q.y$\ $\LTHEN$\ $q.y$ - $p.y$\ $\LELSE$\ $p.y$ - $q.y$\ $\LFI$
    }

    %\vspace{2mm}
    \tadAxioma{$d$ $\cup_{dicc}$ $d'$}{
        $\LIF$ vacio?(claves($d'$)) $\LTHEN$ \\
            \tab $d$ \\
        $\LELSE$ $\LIF$ $\lnot$ def?(proximo, $d$) $\LTHEN$ \\
            \tab definir(proximo, obtener(proximo, $d'$), $d$) $\cup_{dicc}$ borrar(proximo, $d'$) \\
        $\LELSE$ \\
            \tab $d$ $\cup_{dicc}$ borrar(proximo, $d'$) \\
        $\LFI$
        \\ donde proximo $\equiv$ dameUno(claves($d'$)) 
    }

    
    %Quiero eliminar las claves que aparecen en ambos dicc en cs'
    %\vspace{2mm}
    \tadAxioma{sacarRepes($cs,\ cs'$)}{
        $\LIF$ vacio?(claves($cs$)) $\LTHEN$ \\
            \tab $cs'$ \\
        $\LELSE$ $\LIF$ def?(proximo, $cs'$) $\LTHEN$ \\ %Hay repetido
                \tab sacarRepes(borrar(proximo, $cs$), borrar(proximo, $cs'$)) \\
        $\LELSE$ \\
            \tab sacarRepes(borrar(proximo, $cs$),  $cs'$) \\
        $\LFI$
        \\ donde proximo $\equiv$ dameUno(claves($cs$))
    }

    \pie{
        \vfill{}
        \item definido en el apartado Definiciones Auxiliares de SimCity.
        \item las funciones $\mathbf{agCasas}$ y $\mathbf{agComercios}$ agregan respectivamente al diccionario de casas/comercios las construcciones 
        de entrada sin importar si esas posiciones ya se encontraban ocupadas o no. Esto no genera un problema en avanzar turno, por sus 
        restricciones, pero si al unir SimCitys (ya que podrian haber colisiones). Para solucionar esto, sacarRepes quita del diccionario
        de entrada las posiciones ocupadas por construcciones ya establecidas. Es decir, como resolución al conflicto de
        colisiones, las construcciones que permanecen son las del SimCity original.
        
    }

    \tadAxioma{avanzarNivel(cs)}{
        $\LIF$ vacio?(claves($cs$)) $\LTHEN$ \\
            \tab $cs$ \\
        $\LELSE$ \\
            \tab definir(dameUno(claves($cs$)), \\
            \tab\tab obtener(dameUno(claves($cs$)), $cs$) + 1, \\
            \tab\tab avanzarNivel(borrar(dameUno(claves($cs$))), $cs$)) \\
        $\LFI$
    }
    \vspace{1mm}

    \tadAxioma{unirConst($cs,\ cn,\ cn'$)}{
        $\LIF$ vacio?($cn'$) $\LTHEN$ \\
            \tab $cn$ \\
        $\LELSE$ $\LIF$ $\lnot$ def?(proximo, $cs$) $\LTHEN$ \\
            \tab unirConst($cs$, \\
                \tab\tab\tab definir(proximo, obtener(proximo, $cn'$), $cn$), \\
                \tab\tab\tab borrar(proximo, $cn'$))    \\         
        $\LELSE$ \\
            \tab unirConst($cs$, $cn$, borrar(proximo, $cn'$)) \\
        $\LFI$
        \\ donde proximo $\equiv$ dameUno(claves($cn'$)) 
    }\vspace{1mm}

    \tadAxioma{manhatizar($cn$, $cn'$)}{
        $\LIF$ vacio?($cn'$) $\LTHEN$ \\
            \tab vacio \\
        $\LELSE$ \\
            \tab definir(proximo, \\
            \tab\tab nivelComercio(proximo, obtener(proximo, $cn'$), $cn$), \\
            \tab\tab manhatizar($cn$, borrar(proximo, $cn'$))) \\
            %nivelComercio devuelve 1 o el nivel más alto de una casa a distancia manhattan <= 3
        $\LFI$
        \\ donde proximo $\equiv$ dameUno(claves($cn'$)) 
    }

    
\end{tad}  

%\newpage
\Titulo{Definiciones Auxiliares de SimCity}
\vspace{3mm}

\tadOperacion{avanzarValido}{SimCity/s,dicc(Pos,Construcción)/cs}{boolean}{}
\tadAxioma{avanzarValido($s,\ cs$)}{
    $\lnot$ vacio?(claves($cs$)) $\wedge$ $\\$ 
    ($\forall$ $p$ : Pos) (def?($p$, $cs$) $\impluego$ $\\$
        \tab ($\lnot$ $p$ $\in$ claves(construcciones($s$))\ $\wedge$ $\\$
        \tab $\lnot$ esRio($p$, mapa($s$)) $\wedge$ $\\$
        \tab (obtener($p$, $cs$) = $"casa"$ $\vee$ obtener($p$, $cs$) = $"comercio"$)) $\\$
    )
}

\vspace{3mm}

\tadOperacion{unirValido}{Simcity/a,SimCity/b}{boolean}{}
\tadAxioma{unirValido($a,\ b$)}{
    ($\forall$ $p$ : Pos)(def?($p$, construcciones($a$)) $\impluego$ $\\$
        \tab $\lnot$ esRio($p$,\ mapa($b$)) $\wedge$ $\\$        %$\tab$ (p \in masNivel(a) \implies \lnot p \in construcc(b))) $\\$
        \tab ($\nexists$\ $otra$ : Pos)$^{1}$(def?($otra$, construcciones($a$))\ $\yluego$ $\\$
        \tab\tab obtener($otra$, construcciones($a$))\ >\ obtener($p$, construcciones($a$))\ $\\$ 
        \tab ) $\impluego$ $\lnot$\ def?($p$, construcciones($b$)) $\\$
    )\ $\wedge$ $\\$
    ($\forall$ $p$ : Pos)(def?($p$, construcciones($b$)) $\impluego$ $\\$
        \tab $\lnot$ esRio($p$,\ mapa($a$)) $\wedge$ $\\$        %$\tab$ (p \in masNivel(a) \implies \lnot p \in construcc(b))) $\\$
        \tab ($\nexists$\ $otra$ : Pos)$^{1}$(def?($otra$, construcciones($b$))\ $\yluego$ $\\$
        \tab\tab obtener($otra$, construcciones($b$))\ >\ obtener($p$, construcciones($b$))\ $\\$
        \tab ) $\impluego$ $\lnot$\ def?($p$, construcciones($a$)) $\\$
    )
}

\pie{
    \vfill{}
    \item Si en la posición hay una construcción de nivel máximo, no puede colisionar con una construcción del otro SimCity.
}
