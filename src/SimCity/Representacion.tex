\begin{Representacion}
  
% descripcion
Un $SimCity$ se compone por la $ubicacion$ y $nivel$ de una serie de $construcciones$, de tipo $Casa$ o $Comercio$, sobre un $Mapa$, y de una $Popularidad$ respecto a la cantidad de uniones que lo modificaron. \\

La $antiguedad$ o $turno$ de la partida se representará a través de un Natural. A su vez, se guardará un $turno$ interno, representativo de la antiguedad 'original' del simCity, si este no perteneciera a ninguna unión. 

La ubicación de las casas y comercios se representará sobre una secuencia de punteros a diccionarios lineales con clave $Pos$ y significado $Construccion$. Bajo esta representación, el $nivel$ de cada casa es el $turno$ interno actual del SimCity menos la posición del diccionario donde se encuentra definida en la secuencia. Esto se debe a que en cada turno se agrega un diccionario nuevo. Similarmente, el $nivel$ de los comercios se representará igual, pero en este caso también se deberá calcular a partir del $nivel$ de las casas a distancia  manhattan $\leq$ 3. 

La $popularidad$ se representará a través de un Natural. El mapa será de tipo $Mapa$ y las uniones se representarán a través de una $lista$ de $Hijos$ que contiene punteros a los $SimCities$ unidos e información relevante para calcular el nivel de sus construcciones. Ya que, una vez unido a otro, un $SimCity$ debe permanecer sin modificar. 

Cabe recordar que para la resolución de colisiones en las uniones, tiene prioridad el nodo izquierdo. 

% estructura
    \begin{Estructura}{SimCity}[estr]
        \begin{Tupla}[estr]
            \tupItem{antiguedad}{Nat}
            \tupItem{turno\ }{Nat}%
            \tupItem{popularidad\ }{Nat}%
            \tupItem{mapa\ }{Mapa}%
            \tupItem{\\construcciones\ }{lista(puntero(diccLineal(Pos, Construcción)))}
            \tupItem{\\uniones\ }{lista(hijo)}
        \end{Tupla}
        
        \vspace{2mm}
        \begin{Tupla}[hijo]
            \tupItem{sc\ }{puntero(estr)}
            \tupItem{\\ turnoUnido\ }{Nat}
        \end{Tupla}
        
        \vspace{2mm}
        \begin{Tupla}[pos]
            \tupItem{x\ }{Nat}
            \tupItem{y\ }{Nat}
        \end{Tupla}

    \end{Estructura}

    %\newpage
% representacion
    \Rep[estr$^{1}$]{$( \\
        %
        \tab (\forall\  h:\ $hijo$)($esta?$(h,\ e.uniones)\ \impluego\ \\
        \tab \tab (h.sc\ \neq\ $NULL\ $\yluego\ rep(*(h.sc)))^{2} \yluego\ \\ %\land\ h.sc\ \notin\ $unirPunteros$($remover$(p,\ e.uniones))
        \tab \tab (e.turno\ \geq\ h.turnoUnido)^{3} \land\ \\
        \tab \tab (\forall\  h':\ $hijo$)($esta?$(h',\ e.uniones)\ \yluego\ $pos$(h',\ e.uniones)\ >\ $pos$(h,\ e.uniones)\ \impluego\ \\
        \tab \tab \tab h'.turnoUnido\ \leq\ h.turnoUnido\ \\
        \tab \tab )^{4}\\
        \tab )\ \yluego\ \\
        %
        \tab (\forall\ p:\ $puntero(dicc(Pos,\ Construccion))$)($esta?$(p,\ e.construcciones)\ \impluego \\
        \tab \tab \neg vacio?(*p)\ \land\ \\
        \tab \tab (\nexists\ p':\ $puntero(dicc(Pos,\ Construccion))$)($esta?$(p',\ e.construcciones)\ \yluego\\ 
        \tab \tab \tab $pos$(p,\ e.construcciones)\ \neq\ $pos$(p'\, e.construcciones)\ \yluego \\
        \tab \tab \tab $claves$(*(p))\ \cap\ $claves$(*(p'))\ \neq\ \emptyset \\
        \tab \tab) \\
        \tab)^{5}\ \yluego \\
        %
        \tab ($long$(e.construcciones)\ =\ e.turno)^{6}\ \yluego \\ 
        %
        \tab (\&e\ \notin\ $Unidos$)^{7}\yluego\ \\
        \tab (\forall\ p:\ $puntero(estr)$)(p\ \in\ $Unidos$\ \impluego\\
        \tab \tab e.antiguedad\ \geq\ (*p).antiguedad \\
        \tab )^{8} \land \\      
        %
        \tab (e.popularidad\ =\ \#($Unidos$))^{9} \land\ \\
        %
        \tab (\forall\ p:\ $Pos$)(p\ \in\ $Casas$\ \impluego \\
        \tab \tab \neg $esRio$(p,\ $Mapas$)\\ %\land\ ($nivel$(e,\ p)\ \leq\ e.antiguedad)\ \\
        \tab )^{10}\ \land \\
        %
        \tab (\forall\ p:\ $Pos$)(p\ \in\ $Comercios$\ \impluego \\
        \tab \tab \neg $esRio$(p,\ $Mapas$)\\ %\land\ ($nivel$(e,\ p)\ \leq\ e.antiguedad)\ \\
        \tab )^{11}\ \land \\
        %
        %\tab (\nexists\ p:\ $Pos$)((p\ \in\ Comercios\ \vee p \in Casas)\ \yluego nivel(e,\ p)\ =\ e.antiguedad)^{12}\ \land\\ 
        \tab $unionesValidas$(e,\ e.uniones)^{13} \\
        )$}
    \vspace{2mm}

    \newpage

    donde 
    \tadAlinearAxiomas{\tab Comercios}
    \tadAxioma{\tab Unidos}{unirPunteros$(e.uniones)$}
    \tadAxioma{\tab Casas}{unirCasas(Ag(\&$e$,\ Unidos))}
    \tadAxioma{\tab Comercios}{unirComercios(Ag(\&$e$,\ Unidos))}
    \tadAxioma{\tab Mapas}{unirMapas(Ag(\&$e$,\ Unidos))}
    \tadNoAlinearAxiomas

    ~

% abstracción
    \AbsFc[estr]{SimCity}{
        $sc:\ $SimCity$\ |\ \\
        \tab $mapa$(sc)\ \igobs\ $Mapas$\ \land\ \\ 
        \tab $casas$(sc)\ \igobs\ $nivelar(Casas)$\ \land\ \\
        \tab $comercios$(sc)\ \igobs\ $nivelar(Comercios)$\ \land\ \\
        \tab $popularidad$(sc)\ \igobs\ e.popularidad \\
        $
    }
    %\vspace{2mm}
    donde 
    \tadAlinearAxiomas{\tab Comercios}
    \tadAxioma{\tab Unidos}{unirPunteros($e.uniones$)}
    \tadAxioma{\tab Casas}{unirCasas(Ag($\&e$,\ Unidos))}
    \tadAxioma{\tab Comercios}{unirComercios(Ag($\&e$,\ Unidos))}
    \tadAxioma{\tab Mapas}{unirMapas(Ag($\&e$,\ Unidos))}
    \tadNoAlinearAxiomas

\end{Representacion}




% auxiliares
\vspace{5mm}
\Titulo{Definiciones Auxiliares}
\vspace{3mm}

Los siguientes reemplazos sintácticos están confinados al contexto del invariante de representación y la función de abstracción del SimCity. En éste sentido, se considera restricción implícita, para cada uno, ser evaluado en un estado que satisfaga parcialmente la representación,-en términos de lógica ternaria-, al momento de 'llamada' dentro de la misma.  

~

\tadOperacion
{unirPunteros}{secu(hijo) /s}{conj(puntero(estr))}{$(\forall h:$ hijo)($h\ \in s\ \impluego\ h.sc\ \neq\ $NULL)}
\textcolor{gray}{// crea un conjunto de punteros a todos los simcities en la unión.}
\tadAxioma
{unirPunteros($s$)}{ 
    \_unirPunteros($s$,\ $\emptyset$)
}

\vspace{4mm}
\tadOperacion
{\_unirPunteros}{secu(hijo) /s, conj(puntero(estr)) /ps}{conj(puntero(estr))}{eq. unirPunteros}%{$(\forall h:$ hijo)($h\ \in s\ \impluego\ h.sc\ \neq\ $NULL)}
\tadAxioma
{\_unirPunteros($s$,\ $ps$)}
{$   
    \LIF\ $vacia?$(s)\ \LTHEN \\                 
    \tab\ ps \\
    \LELSE\ \LIF\ $prim$(s).sc\ \in\ ps\ \LTHEN\ $\textcolor{gray}{\tab// por si hay loops}$ \\  
    \tab $\_unirPunteros$($fin$(s),\ ps) \\
    \LELSE \\
    \tab $\_unirPunteros$((*($prim$(s).sc)).uniones,\ $Ag$($prim$(s).sc,\ ps))\ \cup\\ 
    \tab \tab $\_unirPunteros$($fin$(s),\ $Ag$($prim$(s).sc,\ ps)) \\
    $\textcolor{gray}{\tab// al unir se descarta el duplicado}$\\
    \LFI
$}

\pie{
    \vfill
    %\setcounter{enumi}{8}
    \item Se asume el traspaso de toda estructura de representación a su equivalente abstracto 
    (se aplica el sombrerito).
    \item Cada hijo apunta a un Simcity válido.
    \item El turno es mayor o igual al turno de cualquier unión inmediata.
    \item Las uniones están ordenadas de más antiguas a más recientes.
    \item Las construcciones en éste nivel no se solapan.
    \item Se agregó al menos un conjunto de construcciones por cada turno interno.
    \item La estructura no loopea consigo misma.
    \item La antiguedad es mayor o igual a la antiguedad de cualquier simCity hijo.
    \item La popularidad es igual a la cantidad de uniones total.
    \item Ninguna casa está sobre un río perteneciente a cualquier mapa en la unión.
    \item Ninguna construcción está sobre un río perteneciente a cualquier mapa en la unión .
    %\item Existe almenos una casa o comercio cuyo nivel es máximo, en el sentido que corresponde a la antiguedad de la partida.
    \item No se solapan posiciones máximas entre esta estructura hasta el hijo 'x', y ese hijo, para todo hijo. Notar que no importan los turnos posteriores al momento de union, porque cualquier construcción agregada posteriormente no podrá ser máxima o modificar los máximos al momento de la unión. Dadas las condiciones ya establecidas hasta este momento.
}

\vspace{4mm}
\tadOperacion
{unirMapas}{conj(puntero(estr)) /ps}{Mapa}{$(\forall p:$ puntero(estr))($p\ \in ps\ \impluego\ (p\ \neq\ $NULL $\yluego$ rep(*p)))}            
\tadAxioma
{unirMapas($ps$)}{$
    \LIF\ $vacio?$(ps)\ \LTHEN\\ 
    \tab $crear$(\emptyset,\ \emptyset)\ \\
    \LELSE\ \\ 
    \tab (*($dameUno$(ps))).mapa\ +\ $UnirMapas(sinUno$(ps))\ \\ 
    \LFI
$}


\vspace{4mm}
\tadOperacion
{unirCasas}{conj(puntero(estr)) /ps}{dicc(Pos $\times$ Nat)}{eq. unirMapas}          
\textcolor{gray}{// crea el diccionario de casas en la unión cuyo significado es igual al turno individual en \\ \tab que se agregó la casa a uno de los simCities en particular.}
\tadAxioma
{unirCasas(ps)}{$
    \LIF\ $vacio?$(ps)\ \LTHEN\ \\ 
    \tab $vacio$\ \\
    \LELSE\ \\
    \tab it((*(dameUno(ps))).construcciones,\ "casa")\ \cup_{dicc}\ $unirCasas(sinUno$(ps))\ \\
    \tab $\textcolor{gray}{// notar que $\cup_{dicc}$ prioriza el argumento izquierdo.}$ \\
    \LFI
$}

\vspace{4mm}
\tadOperacion
{unirComercios}{conj(puntero(estr)) /ps}{dicc(Pos $\times$ Nat)}{eq. unirMapas} 
\textcolor{gray}{// crea el diccionario de comercio en la unión cuyo significado es igual al turno individual \\ \tab en que se agregó la casa a uno de los simCities en particular.}
\tadAxioma
{unirComercios($ps$)}{$
    \LIF\ $vacio?$(ps)\ \LTHEN\ \\
    \tab $vacio$\ \\
    \LELSE\ \\
    \tab it((*(dameUno(ps))).construcciones,\ "comercio")\ \cup_{dicc}\ $unirComercios(sinUno$(ps))\ \\
    \LFI
$}

\vspace{4mm}
\tadOperacion
{it}{secu(puntero(dicc(Pos, Construccion))) /s, Construccion /c}{dicc(Pos\ $\times$ Nat)}{}
\textcolor{gray}{// el significado representa el turno en que se agregó la construcción.}  
\tadAxioma
{it($s$, $c$)}{
    $\_$it($s$, $c$, 0)
}

\vspace{4mm}
\tadOperacion
{\_it}{secu(puntero(dicc(Pos, Construccion))) /s, Construccion /c, Nat /n}{dicc(Pos\ $\times$ Nat)}{}
\tadAxioma
{\_it($s$, $c$, $n$)}{
    $\LIF$ vacia?($s$) $\LTHEN$ 
        vacio 
    $\LELSE$ 
        filtrar(*(prim($s$)), $c$, $n$) $\cup_{dicc}$ it(fin($s$), $c$, $n$ + 1) 
    $\LFI$
}

\vspace{4mm}
\tadOperacion
{filtrar}{dicc(Pos, Construccion), Construccion, Nat }{dicc(Pos\ $\times$ Nat)}{}
\tadAxioma
{filtrar($d$, $c$, $n$)}{
    $\LIF$ vacio?($d$) $\LTHEN$ \\ 
    \tab vacio \\
    $\LELSE$ $\LIF$ obtener(clave, $d$) = $c$ $\LTHEN$ \\ 
    \tab definir(clave, $n$, filtrar(borrar(clave, $d$), $c$, $n$)) \\
    $\LELSE$ \\
    \tab filtrar(borrar(clave, $d$), $c$, $n$) \\
    $\LFI$
}
donde 
\tadAxioma{clave}{dameUno(claves($d$))}

%\vspace{4mm}
%\tadOperacionInline
%{remover}{secu($\alpha$) /s, $\alpha$ /a}{secu($\alpha$)}{} \textcolor{gray}{\tab // remueve la primer aparición, si hay}                      
%\tadAxioma
%{remover($s$,\ $a$)}{$
%    \LIF\ $vacia?$(s)\ \LTHEN\ \\
%    \tab <> \\ 
%    \LELSE\ \LIF\ 
%        a\ =\ $prim$(s)\ 
%    \LTHEN\ \\
%    \tab $fin$(s)\ \\
%    \LELSE\ \\
%    \tab  $prim$(s)\ \bullet\ $remover(fin$(s),\ a)\ \\
%    \LFI
%$}
\vspace{4mm}
\tadOperacion
{pos}{secu($\alpha$) /s,\ $\alpha$ /a}{Nat}{esta?($a$, $s$)}
\tadAxioma
{pos($s$, $a$)}{
    $\LIF$ prim($s$) = $a$ $\LTHEN$ 0 $\LELSE$ 1 + pos(fin($s$), $a$) $\LFI$ 
}

\vspace{4mm}
\tadOperacion
{unionesValidas}{estr /e, secu(hijo) /s}{bool}{$s \subseteq e.uniones$ $\yluego$ $(\forall h:$ hijo)$(h\ \in\ s\ \impluego\ h.sc\ \neq\ $NULL$\ \yluego\ $rep$(*(h.sc))$)}            
\tadAxioma
{unionesValidas($e$,\ $s$)}{$
    $vacio?$(s)\ \oluego\ ($maxcons$(e,\ $izq$)\ 
    \cap\ $maxcons$(e,\ $der$)\ =\ \emptyset\ \land\ $unionesValidas$(e,\ $com$(s)))
$}
donde 
\tadAlinearAxiomas{\tab comercom}
\tadAxioma{\tab com}{unirPunteros(com$(s))$}
\tadAxioma{\tab ult}{ult($s)\ \bullet\ <>$}
\tadAxioma{\tab casascom}{unirCasas(com)} %$\ \cup_{dicc}$\ filtrar($e.casas$,\ ult($s).turnosDesdeUnion)^{1}$}
\tadAxioma{\tab comercom}{unirComercios(com)}%\ $\cup_{dicc}$\ filtrar($e.comercios$,\ ult($s).turnosDesdeUnion)^{1}$}
\tadAxioma{\tab casasult}{unirCasas(ult)}
\tadAxioma{\tab comerult}{unirComercios(ult)}
\tadAxioma{\tab izq}{claves(casascom)\ $\cup$\ claves(comercom)}
\tadAxioma{\tab der}{claves(casasult)\ $\cup$\ claves(comerult)}   
\tadNoAlinearAxiomas


%\vspace{4mm}
%\tadOperacion
%{filtrar}{dicc(Pos, Nat) /d, Nat /n}{dicc(Pos $\times$ Nat)}{}            
%\tadAxioma
%{filtrar($d$,\ $n$)}{$
%    \LIF\ $vacio?$(d)\ \LTHEN\\ 
%    \tab $vacio$\\
%    \LELSE\ \LIF\ $sig$\ \leq\ n\ \LTHEN\\
%    \tab $filtrar(borrar(clave$,\ d),\ n) \\
%    \LELSE \\
%    \tab $definir(clave,\ obtener(clave,\ $d$),\ filtrar(borrar(clave,$\ d),\ n)) \\
%    \LFI
%$}
%donde
%\tadAlinearAxiomas{\tab clave}
%\tadAxioma{\tab clave}{dameUno(claves($d))$}
%\tadNoAlinearAxiomas

%\newpage

\vspace{4mm}
\tadOperacion
{maxcons}{estr /e, conj(Pos) /c}{conj(Pos)}{($\forall p:\ $Pos$)(p\ \in\ c\ \impluego$ $p\ \in\ $ posiciones($e))$}
\tadAxioma
{maxcons($e$,\ $c$)}{$
    $\_maxcons$(e,\ c,\ \emptyset,\ 0)
$}


\vspace{4mm}
\tadOperacion
{\_maxcons}{estr /e, conj(Pos) /c, conj(Pos) /max, Nat /n}{conj(Pos)}{eq. maxcons}          
\tadAxioma
{\_maxcons($e$,\ $c$,\ $max$,\ $n$)}{$
    \LIF\ $vacio?$(c)\ \LTHEN\ \\  
    \tab max \\
    \LELSE\ \LIF\ $nivel$_i\ >\ n\ \LTHEN\ \\ 
    \tab $\_maxcons$(e,\ $sinUno$(c),\ $Ag(pos$_i,\ \emptyset),\ $nivel$_i) \\
    \LELSE\ \LIF\ $nivel$_i\ =\ n\ \LTHEN\ \\ 
    \tab $\_maxcons$(e,\ $sinUno$(c),\ $Ag(pos$_i,\ max),\ n) \\
    \LELSE\ \\
    \tab $\_maxcons$(e,\ $sinUno$(c),\ max,\ n) \\
    \LFI  
$}

%\pie{
%    \vfill{}
%    \item las casas o comercios de éste simCity particular con nivel o nivel base $\leq$ turnosDesdeUnion son aquellas que se agregaron después de la unión.
%}

donde
\tadAlinearAxiomas{\tab nivel$_i$}
\tadAxioma{\tab pos$_i$}{dameUno($c)$}
\tadAxioma{\tab nivel$_i$}{nivel($e$,\ pos$_i$)}
\tadNoAlinearAxiomas
    
\vspace{4mm}
\tadOperacion
{nivel}{estr /e, Pos /pos}{Nat}{$p\ \in\ $ posiciones($e)$}%{pos $\in$ Casas $\vee$ pos $\in$ Comercios}
\textcolor{gray}{// calcula el nivel propiamente de una construcción.}
\tadAxioma
{nivel($e$,\ $pos$)}{$
    \LIF\ $def?$(pos,\ $Casas$)\ \LTHEN\ \\
    \tab $nivelesPorUnion$(e,\ pos) \\
    \LELSE \\ 
    \tab $max(nivelesPorUnion$(e,\ pos),\ $nManhattan)\\\
    $\LFI$
}
donde:
\tadAlinearAxiomas{\tab nManhattan}
\tadAxioma{\tab Unidos}{unirPunteros($e.uniones)$}
\tadAxioma{\tab Casas}{unirCasas(Ag($\&e$,\ Unidos))}
\tadAxioma{\tab Comercios}{unirComercios(Ag($\&e$,\ Unidos))}
\tadAxioma{\tab nManhattan}{manhattan($e$, $pos$,\ Casas)}
\tadNoAlinearAxiomas

\vspace{4mm}
\tadOperacion
{nivelesPorUnion}{estr /e, Pos /pos}{Nat}{eq. nivel}            
\tadAxioma
{nivelesPorUnion($e$,\ $pos$)}{$
    \LIF\ $def?$(pos,\ $Construcciones$)\ \LTHEN\ \\
    \tab e.turno\ -\ $obtener($pos$, Construcciones)$\\
    \LELSE\ \\
    \tab e.turno\ -\ $hijoCorrecto$.turnoUnion\ +\ $nivelesPorUnion(hijoCorrecto$.sc,\ pos) \\
    \LFI 
$}
donde:
\tadAxioma{\tab Construcciones}{it($e.construcciones$, $"casa"$)\ $\cup_{dicc}$\ it($e.construcciones$, $"comercio"$)\ }
\tadAxioma{\tab hijoCorrecto}{llegar($e.uniones,\ pos)$}
    
\vspace{4mm}
\tadOperacion
{llegar}{secu(hijo) /s, Pos /p}{hijo}{($\forall$ $h:$\ hijo)(esta?($h,\ s)\ \impluego\ h.sc\ \neq\ $NULL$\ \yluego\ $rep($*(h.sc)))\ \yluego\ $\\$ (\exists h:$\ hijo)(esta?($h$, $s)\ \yluego\ p\ \in\ posiciones(h.sc))$} 
\textcolor{gray}{// busca el SimCity inmediato en el cual está definida una posición.}
\tadAxioma
{llegar($s$,\ $p$)}{$
    \LIF\ $def?$(pos,\ $unirCasas$(hijo_i))\ \vee\ $def?$(pos,\ $unirComercios$(hijo_i))\ \LTHEN\ \\ 
    \tab $prim$(s) \\ 
    \LELSE\ \\
    \tab $llegar(fin$(s)) \\ 
    \LFI
$}
donde
\tadAxioma{\tab hijo$_i$}{Ag(prim($s$).$sc$,\ $\emptyset)$}

\newpage

\vspace{4mm}
\tadOperacion
{manhattan}{estr /e, Pos /p, Dicc(Pos, Nat) /d}{Conj(Pos)}{($\forall p': $Pos$)(p'\ \in\ $claves$(d)\ \impluego\ p'\ \in\ posiciones(e)$)}            
\tadAxioma
{manhattan($e$, $p$, $d$)}{
    $\LIF$\ vacio?(claves($d$)) $\LTHEN$ \\
        \tab 0 \\
    $\LELSE$ $\LIF$ distManhattan($p$, proximo) $\leq$ 3 $\land$ $p$ $\neq$ proximo $\LTHEN$ \\
        \tab max(nivel($e$, proximo)), manhattan($p$, borrar(proximo, $d$)) \\
    $\LELSE$ \\
        \tab manhattan($e$, $p$, borrar(proximo, $d$)) \\
    $\LFI$  
}
donde 
\tadAxioma
{\tab proximo}{dameUno(claves($d$))}  

\vspace{4mm}
\tadOperacion
{posiciones}{estr /e}{dicc(Pos $\times$ Nat)}{($\forall h:$\ hijo)($h\ \in\ e.uniones\ \impluego\ h.sc\ \neq\ $NULL$\ \yluego\ $rep($*(h.sc)))$}
\tadAxioma
{posiciones(e)}{
    claves(unirCasas(Ag($\&e$, unirPunteros($e.uniones$)))) $\cup$ $\\$ claves(unirComercios(Ag($\&e$, unirPunteros($e.uniones$))))
}
\vspace{4mm}
\tadOperacion
{nivelar}{estr /e, dicc(Pos, Nat) /d}{dicc(Pos $\times$ Nat)}{($\forall p: $Pos$)(p\ \in\ $claves$(d)\ \impluego\ p\ \in\ posiciones(e)$)}            
\tadAxioma
{nivelar($e$, $d$)}{
    $\LIF\ $vacio?$(d)\ \LTHEN\ $vacio$\ \LELSE\ $definir(clave,\ nivel($e$,\ clave),\ nivelar($e$,\ borrar(clave,\ $d)))\ \LFI$
}
donde
\tadAxioma{\tab clave}{dameUno(claves($d))$}