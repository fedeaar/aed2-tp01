\begin{Representacion}
  
% descripcion
Un $SimCity$ se compone por la $ubicacion$ y $nivel$ de una serie de $construcciones$, de tipo $Casa$ o $Comercio$, sobre un $Mapa$, y de una $Popularidad$ respecto a la cantidad de uniones que lo modificaron. \\

La ubicación de las casas se representan sobre un diccionario lineal con clave $Pos$ y significado $Nivel$. La ubicación de los comercios se representan similarmente, pero su significado responde a un $NivelBase$ de tipo $Nat$ a partir del cual se calcula propiamente su $Nivel$. El mapa es de tipo $Mapa$ y las uniones se representan a través de una $lista$ de $Hijos$ que contiene punteros a los $SimCities$ unidos e información relevante para calcular el nivel de sus construcciones. Ya que, una vez unido a otro, un $SimCity$ debe permanecer sin modificar. 

%\vspace{5mm}
% estructura
    \begin{Estructura}{SimCity}[estr]
        \begin{Tupla}[estr]
            \tupItem{turno\ }{Nat}%
            \tupItem{\\popularidad\ }{Nat}%
            \tupItem{\\mapa\ }{Mapa}%
            \tupItem{\\casas\ }{diccLineal(pos,\ Nivel)}
            \tupItem{\\comercios\ }{diccLineal(pos,\ NivelBase)}
            \tupItem{\\uniones\ }{lista(hijo)}
        \end{Tupla}
        
        \vspace{2mm}
        \begin{Tupla}[hijo]
            \tupItem{sc\ }{puntero(estr)}
            \tupItem{\\ turnosDesdeUnion\ }{Nat}
        \end{Tupla}
        
        \vspace{2mm}
        \begin{Tupla}[pos]
            \tupItem{x\ }{Nat}
            \tupItem{y\ }{Nat}
        \end{Tupla}

    \end{Estructura}

% representacion
    \Rep[estr$^{1}$]{$( \\
        \tab (\forall\  h:\ $hijo$)($esta?$(h,\ e.uniones)\ \impluego\ \\
        %11, 12
        \tab \tab h.sc\ \neq\ $NULL\ $\yluego\ rep(*(h.sc))^{2}\ \land\ h.sc\ \notin\ $unirPunteros$($remover$(p,\ e.uniones))^{3}\ \yluego\ \\
        \tab \tab (e.turno\ \geq\ h.turnosDesdeUnion)^{4}\ \land\ \\
        %15
        \tab \tab (\forall\  h_2:\ $hijo$)($esta?$(h_2,\ e.uniones)\ \yluego\ $pos$(h_2,\ e.uniones)\ >\ $pos$(h,\ e.uniones)\ \impluego\ \\
        \tab \tab \tab h_2.turnosDesdeUnion\ \leq\ h.turnosDesdeUnion\ \\
        \tab \tab )^{5} \\
        \tab )\ \yluego\ \\
        \tab (\&e\ \notin\ $Unidos$)^{6} \yluego\ \\
        %2
        \tab (e.popularidad\ =\ \#($Unidos$))^{7}\ \land\ \\
        %3
        \tab (\forall\ p:\ $puntero(estr)$)(p\ \in\ $Unidos$\ \impluego\\
        \tab \tab e.turno\ \geq (*p).turno \\
        \tab )^{8}\ \land \\       
        %
        \tab (\forall\ p:\ $Pos$)(p\ \in\ $claves(e.casas)$\ \impluego \\
        %4, 5, 6
        \tab \tab \neg $def$(p,\ e.comercios)^{9}\ \land\ \neg $esRio$(p,\ $Mapas$)^{10}\ \land\ ($obtener$(p,\ $e.casas$)\ <\ e.turno)^{11}\ \\
        \tab )\ \land \\
        %
        \tab (\forall\ p:\ pos)(p\ \in\ $claves(e.comercios)$ \impluego\ \\
        %7, 8, 9
        \tab \tab \neg $def$(p,\ e.casas)^{12}\ \land\ \neg $esRio$(p,\ $Mapas$)^{13}\ \land\ ($obtener$(p,\ $e.comercios$)\ <\ e.turno)^{14}\ \\
        \tab )\ \yluego\ \\
        %10
        %\tab (\forall\  n:\ Nat)(0\ \leq n\ <\ e.turno\ \impluego\ \\
        %\tab \tab (\exists\ p:\ pos)(def?(p,\ Casas)\ \yluego\ obtener(p,\ Casas)\ =\ n)\ \vee\ \\ 
        %\tab \tab (\exists\ p:\ pos)(def?(p,\ Comercios)\ \yluego\ obtener(p,\ Comercios)\ =\ n) \\  
        %\tab )^{10}\ \land\ \\ 
        \tab $unionesValidas$(e,\ e.uniones)^{15} \\
        )$}
    \vspace{2mm}
    donde 
    \tadAlinearAxiomas{\tab Comercios}
    \tadAxioma{\tab Unidos}{unirPunteros$(e.uniones)$}
    %\tadAxioma{\tab Casas}{unirCasas(Ag(\&$e$,\ Unidos))}
    %\tadAxioma{\tab Comercios}{unirComercios(Ag(\&$e$,\ Unidos))}
    \tadAxioma{\tab Mapas}{$e.mapa$ + unirMapas(Unidos)}
    \tadNoAlinearAxiomas

    

\pie{
    \vfill
    \item Se asume el traspaso de toda estructura de representación a su equivalente abstracto 
    (se aplica el sombrerito).
    \item Cada hijo apunta a un Simcity válido.
    \item El puntero de cada hijo no aparece en ningún otro SimCity de la unión.
    \item El turno es mayor o igual a la cantidad de turnos que pasaron desde la unión.
    \item Las uniones están ordenadas de más antiguas a más recientes.
    \item La estructura no loopea consigo misma.
    \item La popularidad es igual a la cantidad de uniones.
    \item El turno actual es mayor o igual al turno de cualquier simCity hijo.
}
%\newpage

% abstracción
    \AbsFc[estr]{SimCity}{
        $sc:\ $SimCity$\ |\ \\
        \tab $mapa$(sc)\ \igobs\ $Mapas$\ \land\ \\ 
        \tab $casas$(sc)\ \igobs\ $nivelar(Casas)$\ \land\ \\
        \tab $comercios$(sc)\ \igobs\ $nivelar(Comercios)$\ \land\ \\
        \tab $popularidad$(sc)\ \igobs\ e.popularidad \\
        $
    }
    %\vspace{2mm}
    donde 
    \tadAlinearAxiomas{\tab Comercios}
    \tadAxioma{\tab Unidos}{unirPunteros($e.uniones$)}
    \tadAxioma{\tab Casas}{unirCasas(Ag($\&e$,\ Unidos))}
    \tadAxioma{\tab Comercios}{unirComercios(Ag($\&e$,\ Unidos))}
    \tadAxioma{\tab Mapas}{unirMapas(Ag($\&e$,\ Unidos))}
    \tadNoAlinearAxiomas

\end{Representacion}

% auxiliares
\vspace{5mm}
\Titulo{Definiciones Auxiliares}
\vspace{3mm}

Los siguientes reemplazos sintácticos están contenidos al contexto del invariante de representación y la función de abstracción del SimCity. En éste sentido, se considera restricción implícita, para cada uno, ser evaluado en un estado que satisfaga parcialmente la representación,-en términos de lógica ternaria-, al momento de 'llamada' dentro de la misma.  

~

\tadOperacion
{unirPunteros}{secu(hijo) /s}{conj(puntero(estr))}{$(\forall h:$ hijo)($h\ \in s\ \impluego\ h.sc\ \neq\ $NULL)}
\tadAxioma
{unirPunteros($s$)}{ 
    \_unirPunteros($s$,\ $\emptyset$)
}

\vspace{4mm}
\tadOperacion
{\_unirPunteros}{secu(hijo) /s, conj(puntero(estr)) /ps}{conj(puntero(estr))}{eq. unirPunteros}%{$(\forall h:$ hijo)($h\ \in s\ \impluego\ h.sc\ \neq\ $NULL)}
\tadAxioma
{\_unirPunteros($s$,\ $ps$)}
{$   
    \LIF\ $vacia?$(s)\ \LTHEN \\                 
    \tab\ ps \\
    \LELSE\ \LIF\ $prim$(s).sc\ \in\ ps\ \LTHEN\ $\textcolor{gray}{\tab// por si hay loops}$ \\  
    \tab $\_unirPunteros$($fin$(s),\ ps) \\
    \LELSE \\
    \tab $\_unirPunteros$((*($prim$(s).sc)).uniones,\ $Ag$($prim$(s).sc,\ ps))\ \cup\\ 
    \tab \tab $\_unirPunteros$($fin$(s),\ $Ag$($prim$(s).sc,\ ps)) \\
    $\textcolor{gray}{\tab// al unir se descarta el duplicado}$\\
    \LFI
$}

\vspace{4mm}
\tadOperacion
{unirMapas}{conj(puntero(estr)) /ps}{Mapa}{$(\forall p:$ puntero(estr))($p\ \in ps\ \impluego\ (p\ \neq\ $NULL $\yluego$ rep(*p)))}            
\tadAxioma
{unirMapas($ps$)}{$
    \LIF\ $vacio?$(ps)\ \LTHEN\\ 
    \tab $crear$(\emptyset,\ \emptyset)\ \\
    \LELSE\ \\ 
    \tab (*($dameUno$(ps))).mapa\ +\ $UnirMapas(sinUno$(ps))\ \\ 
    \LFI
$}

\vspace{4mm}
\tadOperacion
{unirCasas}{conj(puntero(estr)) /ps}{dicc(Pos $\times$ Nivel)}{eq. unirMapas}            
\tadAxioma
{unirCasas(ps)}{$
    \LIF\ $vacio?$(ps)\ \LTHEN\ \\ 
    \tab $vacio$\ \\
    \LELSE\ \\
    \tab (*p).casas\ \cup_{dicc}\ $unirCasas(sinUno$(ps))\ \\
    \LFI
$}

\pie{
    \vfill
    \setcounter{enumi}{8}
    \item Ninguna casa está en la posición de uno de los comercios de este simCity particular.
    \item Ninguna casa está sobre un río perteneciente a cualquier mapa en la unión.
    \item El turno es mas grande que el nivel de cualquier casa.
    \item Ningún comercio está en la posición de una de las casas de este simCity particular.
    \item Ningún comercio en la unión está sobre un río perteneciente a cualquier mapa en la unión.
    \item El turno es mas grande que el nivel base de cualquier comercio.
    % \item se agregó al menos una consrucción en cada turno hasta el actual, en almenos un simCity en la unión.
   
    \item No se solapan posiciones máximas entre esta estructura hasta el hijo 'x', descontando construcciones agregadas después de la unión, y ese hijo, para todo hijo.
}

\vspace{4mm}
\tadOperacion
{unirComercios}{conj(puntero(estr)) /ps}{dicc(Pos $\times$ Nat)}{eq. unirMapas} 
\tadAxioma
{unirComercios($ps$)}{$
    \LIF\ $vacio?$(ps)\ \LTHEN\ \\
    \tab $vacio$\ \\
    \LELSE\ \\
    \tab (*p).comercios\ \cup_{dicc}\ $unirComercios(sinUno$(ps))\ \\
    \LFI
$}


  

\vspace{4mm}
\tadOperacionInline
{remover}{secu($\alpha$) /s, $\alpha$ /a}{secu($\alpha$)}{} \textcolor{gray}{\tab // remueve la primer aparición, si hay}                      
\tadAxioma
{remover($s$,\ $a$)}{$
    \LIF\ $vacia?$(s)\ \LTHEN\ \\
    \tab <> \\ 
    \LELSE\ \LIF\ 
        a\ =\ $prim$(s)\ 
    \LTHEN\ \\
    \tab $fin$(s)\ \\
    \LELSE\ \\
    \tab  $prim$(s)\ \bullet\ $remover(fin$(s),\ a)\ \\
    \LFI
$}

\vspace{4mm}
\tadOperacion
{unionesValidas}{estr /e, secu(hijo) /s}{bool}{$s \subseteq e.uniones$ $\yluego$ $(\forall h:$ hijo)$(h\ \in\ s\ \impluego\ h.sc\ \neq\ $NULL$\ \yluego\ $rep$(*(h.sc))$)}            
\tadAxioma
{unionesValidas($e$,\ $s$)}{$
    $vacio?$(s)\ \oluego\ ($maxcons$(e,\ $izq$)\ 
    \cap\ $maxcons$(e,\ $der$)\ =\ \emptyset\ \land\ $unionesValidas$(e,\ $com$(s)))
$}
donde 
\tadAlinearAxiomas{\tab comercom}
\tadAxioma{\tab com}{unirPunteros(com$(s))$}
\tadAxioma{\tab ult}{ult($s)\ \bullet\ <>$}
\tadAxioma{\tab casascom}{unirCasas(com)\ $\cup_{dicc}$\ filtrar($e.casas$,\ ult($s).turnosDesdeUnion)^{1}$}
\tadAxioma{\tab comercom}{unirComercios(com)\ $\cup_{dicc}$\ filtrar($e.comercios$,\ ult($s).turnosDesdeUnion)^{1}$}
\tadAxioma{\tab casasult}{unirCasas(ult)}
\tadAxioma{\tab comerult}{unirComercios(ult)}
\tadAxioma{\tab izq}{claves(casascom)\ $\cup$\ claves(comercom)}
\tadAxioma{\tab der}{claves(casasult)\ $\cup$\ claves(comerult)}   
\tadNoAlinearAxiomas


\vspace{4mm}
\tadOperacion
{filtrar}{dicc(Pos, Nat) /d, Nat /n}{dicc(Pos $\times$ Nat)}{}            
\tadAxioma
{filtrar($d$,\ $n$)}{$
    \LIF\ $vacio?$(d)\ \LTHEN\\ 
    \tab $vacio$\\
    \LELSE\ \LIF\ $sig$\ \leq\ n\ \LTHEN\\
    \tab $filtrar(borrar(clave$,\ d),\ n) \\
    \LELSE \\
    \tab $definir(clave,\ obtener(clave,\ $d$),\ filtrar(borrar(clave,$\ d),\ n)) \\
    \LFI
$}
donde
\tadAlinearAxiomas{\tab clave}
\tadAxioma{\tab clave}{dameUno(claves($d))$}
\tadNoAlinearAxiomas

\vspace{4mm}
\tadOperacion
{maxcons}{estr /e, conj(Pos) /c}{conj(Pos)}{($\forall p:\ $Pos$)(p\ \in\ c\ \impluego$ $p\ \in\ $ posiciones($e))$}
\tadAxioma
{maxcons($e$,\ $c$)}{$
    $\_maxcons$(e,\ c,\ \emptyset,\ 0)
$}

\vspace{4mm}
\tadOperacion
{\_maxcons}{estr /e, conj(Pos) /c, conj(Pos) /max, Nat /n}{conj(Pos)}{eq. maxcons}          
\tadAxioma
{\_maxcons($e$,\ $c$,\ $max$,\ $n$)}{$
    \LIF\ $vacio?$(c)\ \LTHEN\ \\  
    \tab max \\
    \LELSE\ \LIF\ $nivel$_i\ >\ n\ \LTHEN\ \\ 
    \tab $\_maxcons$(e,\ $sinUno$(c),\ $Ag(pos$_i,\ \emptyset),\ $nivel$_i) \\
    \LELSE\ \LIF\ $nivel$_i\ =\ n\ \LTHEN\ \\ 
    \tab $\_maxcons$(e,\ $sinUno$(c),\ $Ag(pos$_i,\ max),\ n) \\
    \LELSE\ \\
    \tab $\_maxcons$(e,\ $sinUno$(c),\ max,\ n) \\
    \LFI  
$}


\pie{
    \vfill{}
    \item las casas o comercios de éste simCity particular con nivel o nivel base $\leq$ turnosDesdeUnion son aquellas que se agregaron después de la unión.
}

donde
\tadAlinearAxiomas{\tab nivel$_i$}
\tadAxioma{\tab pos$_i$}{dameUno($c)$}
\tadAxioma{\tab nivel$_i$}{nivel($e$,\ pos$_i$)}
\tadNoAlinearAxiomas
    
\vspace{4mm}
\tadOperacion
{nivel}{estr /e, Pos /pos}{Nat}{$p\ \in\ $ posiciones($e)$}%{pos $\in$ Casas $\vee$ pos $\in$ Comercios}
\tadAxioma
{nivel($e$,\ $pos$)}{$
    \LIF\ $def?$(pos,\ $Casas$)\ \LTHEN\ \\
    \tab $obtener$(pos,\ $Casas$)\ +\ 
        $nivelesPorUnion$(e,\ pos) \\
    \LELSE \\ 
    \tab $max(obtener($pos,\ $Comercios$)\ +\ $nivelesPorUnion$(e,\ pos),\ $nManhattan$)\\\
    \LFI
$}
donde:
\tadAlinearAxiomas{\tab nManhattan}
\tadAxioma{\tab Unidos}{unirPunteros($e.uniones)$}
\tadAxioma{\tab Casas}{unirCasas(Ag($\&e$,\ Unidos))}
\tadAxioma{\tab Comercios}{unirComercios(Ag($\&e$,\ Unidos))}
\tadAxioma{\tab nManhattan}{manhattan($e$, $pos$,\ Casas)}
\tadNoAlinearAxiomas

\vspace{4mm}
\tadOperacion
{nivelesPorUnion}{estr /e, Pos /pos}{Nat}{eq. nivel}            
\tadAxioma
{nivelesPorUnion($e$,\ $pos$)}{$
    \LIF\ $def?$(pos,\ e.casas)\ \vee\ $def?$(pos,\ e.comercios)\ \LTHEN\ \\
    \tab 0 \\
    \LELSE\ \\
    \tab $hijoCorrecto$.turnosDesdeUnion\ +\ $nivelesPorUnion(hijoCorrecto$.sc,\ pos) \\
    \LFI 
$}
donde:
\tadAxioma{\tab hijoCorrecto}{llegar($e.uniones,\ pos)$}
    
\vspace{4mm}
\tadOperacion
{llegar}{secu(hijo) /s, Pos /p}{hijo}{($\forall$ $h:$\ hijo)(esta?($h,\ s)\ \impluego\ h.sc\ \neq\ $NULL$\ \yluego\ $rep($*(h.sc)))\ \yluego\ $\\$ (\exists h:$\ hijo)(esta?($h$, $s)\ \yluego\ p\ \in\ posiciones(h.sc))$} 
\tadAxioma
{llegar($s$,\ $p$)}{$
    \LIF\ $def?$(pos,\ $unirCasas$(hijo_i))\ \vee\ $def?$(pos,\ $unirComercios$(hijo_i))\ \LTHEN\ \\ 
    \tab $prim$(s) \\ 
    \LELSE\ \\
    \tab $llegar(fin$(s)) \\ 
    \LFI
$}
donde
\tadAxioma{\tab hijo$_i$}{Ag(prim($s$).$sc$,\ $\emptyset)$}

\vspace{4mm}
\tadOperacion
{manhattan}{estr /e, Pos /p, Dicc(Pos, Nat) /d}{Conj(Pos)}{($\forall p': $Pos$)(p'\ \in\ $claves$(d)\ \impluego\ p'\ \in\ posiciones(e)$)}            
\tadAxioma
{manhattan($e$, $p$, $d$)}{
    $\LIF$\ vacio?(claves($d$)) $\LTHEN$ \\
        \tab 1 \\
    $\LELSE$ $\LIF$ distManhattan($p$, proximo) $\leq$ 3 $\land$ $p$ $\neq$ proximo $\LTHEN$ \\
        \tab max(obtener(proximo, $d$) + nivelesPorUnion($e$, proximo),\\
        \tab\tab \ manhattan($p$, borrar(proximo, $d$))) \\
    $\LELSE$ \\
        \tab manhattan($e$, $p$, borrar(proximo, $d$)) \\
    $\LFI$  
    \\ donde proximo $\equiv$ dameUno(claves($d$)) 
}

\vspace{4mm}
\tadOperacion
{posiciones}{estr /e}{dicc(Pos $\times$ Nat)}{($\forall h:$\ hijo)($h\ \in\ e.uniones\ \impluego\ h.sc\ \neq\ $NULL$\ \yluego\ $rep($*(h.sc)))$}
\tadAxioma
{posiciones(e)}{
    claves(e.casas $\cup_{dicc}$ unirCasas(unirPunteros($e.uniones$))) $\cup$ $\\$ claves(e.comercios $\cup_{dicc}$ unirComercios(unirPunteros($e.uniones$)))
}
\vspace{4mm}
\tadOperacion
{nivelar}{estr /e, dicc(Pos, Nat) /d}{dicc(Pos $\times$ Nat)}{($\forall p: $Pos$)(p\ \in\ $claves$(d)\ \impluego\ p\ \in\ posiciones(e)$)}            
\tadAxioma
{nivelar($e$, $d$)}{
    $\LIF\ $vacio?$(d)\ \LTHEN\ $vacio$\ \LELSE\ $definir(clave,\ nivel($e$,\ clave),\ nivelar($e$,\ borrar(clave,\ $d)))\ \LFI$
}
donde
\tadAxioma{\tab clave}{dameUno(claves($d))$}