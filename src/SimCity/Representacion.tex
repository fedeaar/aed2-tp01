\begin{Representacion}
  
% descripcion
$SimCity$ se compone por la $ubicacion$ y $nivel$ de una serie de $construcciones$, de tipo $casa$ o $comercio$, sobre un $Mapa$, y de una $popularidad$ respecto a la cantidad de uniones que lo modificaron. \\

La ubicación de las casas se representan sobre un diccionario lineal con clave $Pos \equiv tupla<Nat,\ Nat>$ y significado $Nivel \equiv Nat$. La ubicación de los comercios se representan similarmente, pero su significado responde a un $NivelBase \equiv Nat$ a partir del cual se calcula propiamente su $nivel$. El mapa es de tipo $Mapa$ y las uniones se representan a través de una $lista$ que contiene punteros a los $SimCitys$ unidos e información relevante para calcular el nivel de sus construcciones. Ya que, una vez unido a otro, un $SimCity$ debe permanecer sin modificación. 

\vspace{5mm}
% estructura

    \begin{Estructura}{SimCity}[estr]
        \begin{Tupla}[estr]
            \tupItem{turno\ }{Nat}%
            \tupItem{\\popularidad\ }{Nat}%
            \tupItem{\\mapa\ }{Mapa}%
            \tupItem{\\casas\ }{diccLineal(pos,\ Nivel)}
            \tupItem{\\comercios\ }{diccLineal(pos,\ NivelBase)}
            \tupItem{\\uniones\ }{lista(hijo)}
        \end{Tupla}
        
        \vspace{2mm}
        \begin{Tupla}[hijo]
            \tupItem{sc\ }{puntero(estr)}
            \tupItem{\\ turnosDesdeUnion\ }{Nat}
        \end{Tupla}
        
        \vspace{2mm}
        \begin{Tupla}[pos]
            \tupItem{x\ }{Nat}
            \tupItem{y\ }{Nat}
        \end{Tupla}
    \end{Estructura}

\vspace{5mm}
% representacion
    \Rep[estr$^{1}$]{$( \\
        %1
        \tab (\&e\ \notin\ Unidos)^{2} \yluego\ \\
        %2
        \tab (e.popularidad\ =\ \#(Unidos))^{3}\ \land\ \\
        %3
        \tab (\forall\ p:\ puntero(estr))(p\ \in\ Unidos\ \impluego\\
        \tab \tab e.turno\ \geq (*p).turno \\
        \tab )^{4}\ \land \\       
        %
        \tab (\forall\ p:\ pos)(p\ \in\ claves(Casas)\ \impluego \\
        %4, 5, 6
        \tab \tab \neg def(p,\ e.comercios)^{5}\ \land\ \neg esRio(p,\ Mapas)^{6}\ \land\ (obtener(p,\ Casas)\ <\ e.turno)^{7}\ \\
        \tab )\ \land \\
        %
        \tab (\forall\ p:\ pos)(p\ \in\ claves(Comercios) \impluego\ \\
        %7, 8, 9
        \tab \tab \neg def(p,\ e.casas)^{8}\ \land\ \neg esRio(p,\ Mapas)^{9}\ \land\ (obtener(p,\ Comercios)\ <\ e.turno)^{10}\ \\
        \tab )\ \land\ \\
        %10
        %\tab (\forall\  n:\ Nat)(0\ \leq n\ <\ e.turno\ \impluego\ \\
        %\tab \tab (\exists\ p:\ pos)(def?(p,\ Casas)\ \yluego\ obtener(p,\ Casas)\ =\ n)\ \vee\ \\ 
        %\tab \tab (\exists\ p:\ pos)(def?(p,\ Comercios)\ \yluego\ obtener(p,\ Comercios)\ =\ n) \\  
        %\tab )^{10}\ \land\ \\ 
        \tab (\forall\  h:\ hijo)(esta?(h,\ e.uniones)\ \impluego\ \\
        %11, 12
        \tab \tab (h.sc\ \neq\ null\ \yluego\ h.sc\ \notin\ unirPunteros(remover(p,\ e.uniones)))^{11}\ \yluego\ \\
        %13
        \tab \tab rep(*(h.sc))^{12}\ \yluego\ \\
        %14
        \tab \tab (e.turno\ \geq\ h.turnosDesdeUnion)^{13}\ \land\ \\
        %15
        \tab \tab (\forall\  h_2:\ hijo)(esta?(h_2,\ e.uniones)\ \yluego\ pos(h_2,\ e.uniones)\ >\ pos(h,\ e.uniones)\ \impluego\ \\
        \tab \tab \tab h_2.turnosDesdeUnion\ \leq\ h.turnosDesdeUnion\ \\
        \tab \tab )^{14} \\
        \tab )\ \land\ \\
        \tab unionesValidas(e,\ e.uniones)^{15} \\
        )$}
    \vspace{2mm}
    \ \ donde 
    \tadAlinearAxiomas{\tab Comercios}
    \tadAxioma{\tab $Unidos$}{$unirPunteros(e.uniones)$}
    \tadAxioma{\tab $Casas$}{$unirCasas(Ag(\&e,\ Unidos))$}
    \tadAxioma{\tab $Comercios$}{$unirComercios(Ag(\&e,\ Unidos))$}
    \tadAxioma{\tab $Mapas$}{$unirMapas(Ag(\&e,\ Unidos))$}
    \tadNoAlinearAxiomas

\vspace{5mm}
% abstracción
    \AbsFc[estr]{SimCity}{
        $sc:\ SimCity\ |\ \\
        \tab mapa(sc)\ \igobs\ Mapas\ \land\ \\ 
        \tab casas(sc)\ \igobs\ nivelar(Casas)\ \land\ \\
        \tab comercios(sc)\ \igobs\ nivelar(Comercios)\ \land\ \\
        \tab popularidad(sc)\ \igobs\ e.popularidad \\
        $
    }
    \vspace{2mm}
    \ \ donde 
    \tadAlinearAxiomas{\tab Comercios}
    \tadAxioma{\tab $Unidos$}{$unirPunteros(e.uniones)$}
    \tadAxioma{\tab $Casas$}{$unirCasas(Ag(\&e,\ Unidos))$}
    \tadAxioma{\tab $Comercios$}{$unirComercios(Ag(\&e,\ Unidos))$}
    \tadAxioma{\tab $Mapas$}{$unirMapas(Ag(\&e,\ Unidos))$}
    \tadNoAlinearAxiomas

\end{Representacion}

\mbox{}
\vfill

\begin{footnotesize}
\begin{enumerate}
    \item Se asume el traspaso de toda estructura de representación a su equivalente abstracto 
    (se aplica el sombrerito).
    \item la estructura no loopea consigo misma.
    \item el turno actual es mayor o igual al turno de cualquier simCity hijo.
    \item la popularidad es igual a la cantidad de uniones.
    \item ninguna casa en la unión está en la posición de uno de los comercios de este simCity particular.
    \item ninguna casa en la unión está sobre un río perteneciente a cualquier mapa en la unión.
    \item el turno es mas grande que el nivel de cualquier casa en la unión.
    \item ningún comercio en la unión está en la posición de una de las casas de este simCity particular.
    \item ningún comercio en la unión está sobre un río perteneciente a cualquier mapa en la unión.
    \item el turno es mas grande que el nivel base de cualquier comercio en la unión.
    % \item se agregó al menos una consrucción en cada turno hasta el actual, en almenos un simCity en la unión.
    \item Cada hijo apunta a un SimCity y su puntero no aparece en ningún otro SimCity de la unión.
    \item Cada hijo apunta a un Simcity válido.
    \item El turno es mayor o igual a la cantidad de turnos que pasaron desde la unión.
    \item Las uniones están ordenadas de más antiguas a más recientes.
    \item No se solapan posiciones máximas entre esta estructura hasta el hijo 'x', descontando construcciones agregadas después de la unión, y ese hijo, para todo hijo.
\end{enumerate}
\end{footnotesize}


% auxiliares
\noindent\textbf{\Large auxiliares para la representación}
\vspace*{2ex}%

\tadOperacion
{unirPunteros}{secu(hijo)}{conj(puntero(estr))}{}
\tadAxioma
{unirPunteros(s)}{ 
    $\_unirPunteros(s,\ \emptyset)$
}

\vspace{4mm}
\tadOperacion
{\_unirPunteros}{secu(hijo), conj(puntero(estr))}{conj(puntero(estr))}{}
\tadAxioma
{\_unirPunteros(s,\ p)}
{$   
    \LIF\ vacia?(s)\ \LTHEN \\                 
    \tab\ p \\
    \LELSE\ \LIF\ prim(s).sc\ \in\ p\ \LTHEN\ $\tab// por si hay loops$ \\  
    \tab \_unirPunteros(fin(s),\ p) \\
    \LELSE \\
    \tab \_unirPunteros((*(prim(s).sc)).uniones,\ Ag(prim(s).sc,\ p))\ \cup\\ 
    \tab \tab \_unirPunteros(fin(s),\ Ag(prim(s).sc,\ p)) \\
    \LFI
$}

\vspace{4mm}
\tadOperacion
{unirMapas}{conj(puntero(estr))}{Mapa}{}            
\tadAxioma
{unirMapas(ps)}{$
    \LIF\ vacio?(ps)\ \LTHEN\\ 
    \tab crear(\emptyset,\ \emptyset)\ \\
    \LELSE\ \\ 
    \tab (*(dameUno(ps))).mapa\ +\ UnirMapas(sinUno(ps))\ \\ 
    \LFI
$}

\vspace{4mm}
\tadOperacion
{unirCasas}{conj(puntero(estr))}{dicc(Pos,\ Nivel)}{}            
\tadAxioma
{unirCasas(ps)}{$
    \LIF\ vacio?(ps)\ \LTHEN\ \\ 
    \tab vacio\ \\
    \LELSE\ \\
    \tab (*p).casas\ \cup_{dicc}\ unirCasas(sinUno(ps))\ \\
    \LFI
$}

\vspace{4mm}
\tadOperacion
{unirComercios}{conj(puntero(estr))}{dicc(Pos,\ Nat)}{} 
\tadAxioma
{unirComercios(ps)}{$
    \LIF\ vacio?(ps)\ \LTHEN\ \\
    \tab vacio\ \\
    \LELSE\ \\
    \tab (*p).comercios\ \cup_{dicc}\ unirComercios(sinUno(ps))\ \\
    \LFI
$}

\vspace{4mm}
\tadOperacionInline
{remover}{secu($\alpha$), $\alpha$}{secu($\alpha$)}{} \tab // remueve la primer aparición                      
\tadAxioma
{remover(s,\ a)}{$
    \LIF\ vacia?(s)\ \LTHEN\ \\
    \tab <> \\ 
    \LELSE\ \LIF\ 
        a\ =\ prim(s)\ 
    \LTHEN\ \\
    \tab fin(s)\ \\
    \LELSE\ \\
    \tab  prim(s)\ \bullet\ remover(fin(s),\ a)\ \\
    \LFI
$}

\vspace{4mm}
\tadOperacion
{unionesValidas}{estr, secu(hijo)}{bool}{}            
\tadAxioma
{unionesValidas(e,\ s)}{$
    vacio?(s)\ \oluego\ (maxcons(e,\ izq)\ 
    \cap\ maxcons(e,\ der)\ =\ \emptyset\ \land\ unionesValidas(e,\ com(s)))
$}
donde 
\tadAlinearAxiomas{\tab comercom}
\tadAxioma{\tab com}{$unirPunteros(com(s))$}
\tadAxioma{\tab ult}{$ult(s)\ \bullet\ <>$}
\tadAxioma{\tab casascom}{$unirCasas(com)\ \cup_{dicc}\ filtrar(e.casas,\ ult(s).turnosDesdeUnion)^{1}$}
\tadAxioma{\tab comercom}{$unirComercios(com)\ \cup_{dicc}\ filtrar(e.comercios,\ ult(s).turnosDesdeUnion)^{1}$}
\tadAxioma{\tab casasult}{$unirCasas(ult)$}
\tadAxioma{\tab comerult}{$unirComercios(ult)$}
\tadAxioma{\tab izq}{$claves(casascom)\ \cup\ claves(comercom)$}
\tadAxioma{\tab der}{$claves(casasult)\ \cup\ claves(comerult)$}   
\tadNoAlinearAxiomas

\mbox{}
\begin{footnotesize}
\begin{enumerate}
    \item las casas o comercios de éste simCity particular con nivel o nivel base $\leq$ turnosDesdeUnion son aquellas que se agregaron después de la unión.
\end{enumerate}
\end{footnotesize}


\vspace{4mm}
\tadOperacion
{filtrar}{dicc(Pos, Nat), Nat}{dicc(Pos, Nat)}{}            
\tadAxioma
{filtrar(d,\ n)}{$
    \LIF\ vacio?(d)\ \LTHEN\\ 
    \tab vacio \\
    \LELSE\ \LIF\ sig\ \leq\ n\ \LTHEN\\
    \tab filtrar(borrar(clave,\ d),\ n) \\
    \LELSE \\
    \tab definir(clave,\ sig,\ filtrar(borrar(clave,\ d),\ n)) \\
    \LFI
$}
donde
\tadAlinearAxiomas{\tab clave}
\tadAxioma{\tab clave}{$dameUno(claves(d))$}
\tadAxioma{\tab sig}{$obtener(clave,\ d)$}
\tadNoAlinearAxiomas

\vspace{4mm}
\tadOperacion
{maxcons}{estr, conj(Pos)}{conj(Pos)}{}            
\tadAxioma
{maxcons(e,\ c)}{$
    \_maxcons(e,\ c,\ \emptyset,\ 0)
$}

\vspace{4mm}
\tadOperacion
{\_maxcons}{estr, conj(Pos), conj(Pos), Nat}{conj(Pos)}{}            
\tadAxioma
{\_maxcons(e,\ c,\ max,\ n)}{$
    \LIF\ vacio?(c)\ \LTHEN\ \\  
    \tab max \\
    \LELSE\ \LIF\ nivel_i\ >\ n\ \LTHEN\ \\ 
    \tab \_maxcons(e,\ sinUno(c),\ Ag(pos_i,\ \emptyset),\ nivel_i) \\
    \LELSE\ \LIF\ nivel_i\ =\ n\ \LTHEN\ \\ 
    \tab \_maxcons(e,\ sinUno(c),\ Ag(pos_i,\ max),\ n) \\
    \LELSE\ \\
    \tab \_maxcons(e,\ sinUno(c),\ max,\ n) \\
    \LFI  
$}
donde
\tadAlinearAxiomas{\tab nivel$_i$}
\tadAxioma{\tab pos$_i$}{$dameUno(c)$}
\tadAxioma{\tab nivel$_i$}{$nivel(e,\ pos_i)$}
\tadNoAlinearAxiomas

\vspace{4mm}
\tadOperacion
{nivel}{estr, pos}{Nat}{}            
\tadAxioma
{nivel(e,\ pos)}{$
    \LIF\ def?(pos,\ Casas)\ \LTHEN\ \\
    \tab obtener(pos,\ Casas)\ +\ 
        nivelesPorUnion(e,\ pos) \\
    \LELSE \\ 
    \tab maximo(obtener(pos,\ Comercios)\ 
        \bullet\ nManhattan)\ +\ nivelesPorUnion(e,\ pos)\\
    \LFI
$}
donde:
\tadAlinearAxiomas{\tab nManhattan}
\tadAxioma{\tab Unidos}{$unirPunteros(e.uniones)$}
\tadAxioma{\tab Casas}{$unirCasas(Ag(\&e,\ Unidos))$}
\tadAxioma{\tab Comercios}{$unirComercios(Ag(\&e,\ Unidos))$}
\tadAxioma{\tab nManhattan}{$significados(manhattan(pos,\ 3),\ Casas)$}
\tadNoAlinearAxiomas

\vspace{4mm}
\tadOperacion
{nivelesPorUnion}{estr, pos}{Nat}{}            
\tadAxioma
{nivelesPorUnion(e,\ pos)}{$
    \LIF\ def?(pos,\ e.casas)\ \vee\ def?(pos,\ e.comercios)\ \LTHEN\ \\
    \tab 0 \\
    \LELSE\ \\
    \tab hijoCorrecto.turnosDesdeUnion\ +\\
    \tab \tab nivelesPorUnion(hijoCorrecto.sc,\ pos) \\
    \LFI 
$}
donde:
\tadAxioma{\tab hijoCorrecto}{$llegar(e.uniones,\ pos)$}
    
\vspace{4mm}
\tadOperacion
{llegar}{secu(hijo), pos}{hijo}{} % $pos\ \in alguno\ de\ los\ hijos$
\tadAxioma
{llegar(s,\ p)}{$
    \LIF\ def?(pos,\ unirCasas(hijo)\ \vee\ def?(pos,\ unirComercios(hijo))\ \LTHEN\ \\ 
    \tab prim(s) \\ 
    \LELSE\ \\
    \tab llegar(fin(s)) \\ 
    \LFI
$}
donde
\tadAxioma{\tab hijo}{$Ag(prim(s).sc,\ \emptyset)$}

\vspace{4mm}
\tadOperacion
{manhattan}{Pos, Nat}{Conj(Pos)}{}            
\tadAxioma
{manhattan(p,\ dist)}{$
    \LIF\ dist\ =\ 0\ \LTHEN\ \\ 
    \tab Ag(p,\ \emptyset) \\
    \LELSE\ \\
    \tab diagonal(\{p.x,\ p.y\ +\ dist\},\ \{p.x\ +\ dist,\ p.y\})\ \cup \\
    \tab \LIF\ p.x\ -\ dist\ \geq\ 0\ \LTHEN\ \\
    \tab \tab diagonal(\{p.x,\ p.y\ +\ dist\},\ \{p.x \- dist,\ p.y\}) \\ 
    \tab \LELSE\ \\ 
    \tab \tab \emptyset\ \\
    \tab \LFI\ \cup\ \\
    \tab \LIF\ p.y\ -\ dist\ \geq\ 0\ \LTHEN\ \\ 
    \tab \tab diagonal(\{p.x,\ p.y \- dist\},\ \{p.x\ +\ dist,\ p.y\}) \\ 
    \tab \LELSE\ \\
    \tab \tab \emptyset\ \\
    \tab \LFI\ \cup\ \\
    \tab \LIF\ p.x\ -\ dist\ \geq\ 0\ \land\ p.y\ - dist\ \geq\ 0\ \LTHEN\ \\
    \tab \tab diagonal(\{p.x ,\ p.y \- dist\},\ \{p.x \- dist,\ p.y\}) \\ 
    \tab \LELSE\ \\
    \tab \tab \emptyset\ \\
    \tab \LFI\ \cup\ \\
    \tab manhattan(p,\ dist\ -\ 1)\ \\
    \LFI
$}

\vspace{4mm}
\tadOperacion
{diagonal}{pos, pos, Nat\ n}{Conj(pos)}{} %$|p1.x\ -\ p2.x|\ =\ |p1.y\ -\ p2.y|\ \land\ 0 \leq y \leq |p2.y\ -\ p1.y|$
\tadAxioma
{diagonal(d,\ h,\ y)}{$
    \LIF\ y\ =\ |h.y\ -\ d.y|\ \LTHEN\ \\
    \tab Ag(h,\ \emptyset) \\
    \LELSE\ \\
    \tab Ag(caminar(d,\ h,\ y),\ diagonal(d,\ h,\ y\ +\ 1))\\ 
    \LFI
$}

\vspace{4mm}
\tadOperacion
{caminar}{pos, pos, Nat}{pos}{} %{$|p1.x\ -\ p2.x|\ =\ |p1.y\ -\ p2.y|\ \land\ 0\ \leq\ y\ \leq\ |p2.y\ -\ p1.y|$}            
\tadAxioma
{caminar(d,\ h,\ y)}{$
    \LIF\ d.y\ \leq\ h.y\ \LTHEN\ \\ 
    \tab \LIF\ d.x\ \leq\ h.x\ \LTHEN\ \\ 
    \tab \tab \{d.x\ +\ y,\ d.y\ +\ y\} \\
    \tab \LELSE\ \\
    \tab \tab \{d.x\ -\ y,\ d.y\ +\ y\} \\
    \tab \LFI\ \\
    \LELSE\ \\
    \tab \LIF\ d.x\ \leq\ h.x\ \LTHEN\ \\
    \tab \tab \{d.x\ +\ y,\ d.y\ -\ y\} \\
    \tab \LELSE\ \\
    \tab \tab \{d.x\ -\ y,\ d.y\ -\ y\} \\
    \tab \LFI \\
    \LFI 
$}

\vspace{4mm}
\tadOperacion
{significados}{conj($\alpha$), dicc($\alpha$,\ $\beta$)}{secu($\alpha$)}{}            
\tadAxioma
{significados(c,\ d)}{$
    \LIF\ vacio?(c)\ \LTHEN\ \\ 
    \tab <> \\
    \LELSE\ \LIF\ def?(dameUno(c),\ d)\ \LTHEN\ \\ 
    \tab obtener(c,\ d)\ \bullet\ significados(sinUno(c),\ d) \\
    \LELSE\ \\
    \tab significados(sinUno(c),\ d) \\
    \LFI 
$}

\vspace{4mm}
\tadOperacion
{maximo}{secu(Nat)\ a}{Nat}{$long(a)\ >\ 0$}            
\tadAxioma
{maximo(s)}{$
    \LIF\ long(s)\ =\ 1\ \LTHEN\ 
        prim(s)\ 
    \LELSE\ max(prim(s),\ maximo(fin(s)))\ \LFI
$}

\vspace{4mm}
\tadOperacion
{nivelar}{estr, dicc(Pos, Nat)}{dicc(Pos, Nat)}{}

\tadAxioma
{nivelar(d)}{
    $\LIF\ vacio?(d)\ \LTHEN\ vacio\ \LELSE\ definir(clave,\ nivel(e,\ clave),\ nivelar(e,\ borrar(clave,\ d)))\ \LFI$
}
donde
\tadAxioma{\tab clave}{$dameUno(claves(d))$}