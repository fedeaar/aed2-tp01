\begin{Algoritmos} \\
    \noindent Observación: Interpretamos que el \textbf{inout} se refiere a la mutabilidad de un parametro. Se toma por contexto si pasa por copia o referencia.\\
    Sea U $\equiv$ \{u$_1$, u$_2$, ... , u$_n$\}, 
    tal que cada uno de esos u$_i$ representa un SimCity anidado al simcity original.\\
    Llamamos nodos : $ \# $U.\\
    Llamamos sumConstrucciones : $\sum_{i=1}^{nodos}$(copiar(u$_i$.construcciones)).\\
    Llamamos sumMapas : $\sum_{i=1}^{nodos}$(u$_i$.mapa).\\
    \\
    \makebox[\linewidth]{\rule{\textwidth}{0.4pt}}
    \textbf{iniciar}(\textbf{in \ }m: \ Mapa) \ $\longrightarrow$ \ res \ : \ res\\
    1: \  \ res.turno \ $\leftarrow$ \ 0\\
    2: \  \ res.antiguedad \ $\leftarrow$ \ 0\\
    3: \  \ res.popularidad \ $\leftarrow$ \ 0\\
    4: \  \ res.mapa \ $\leftarrow$ \ m\\
    5: \  \ res.construcciones \ $\leftarrow$ \ vacio()\\
    6: \  \ res.uniones \ $\leftarrow$ \ vacia()\\
    7: \  \ \textbf{return} \ res\\
    \textbf{Complejidad}: \ O(copy(m))\\
    \makebox[\linewidth]{\rule{\textwidth}{0.4pt}}
    \\
    \makebox[\linewidth]{\rule{\textwidth}{0.4pt}}
    \textbf{avanzarTurno}(\textbf{inout} \ SimCity \ s, \ \textbf{in \ }dicc(Pos, \ Construccion) \ cs)\\
    1: \  \ s.turno \ $\leftarrow$ \ s.turno \ + \ 1 \ \\
    2: \  \ s.antiguedad \ $\leftarrow$ \ s.antiguedad \ + \ 1 \ \\
    3: \  \ s.construcciones.agregarAtras(direccion(cs))\\
    \textbf{Complejidad}: \ O(1)\\
    \makebox[\linewidth]{\rule{\textwidth}{0.4pt}}
    \\
    \makebox[\linewidth]{\rule{\textwidth}{0.4pt}}
    \textbf{unir}(\textbf{inout} \ SimCity \ s1, \ \textbf{in \ }Simcity \ s2)\\
    1: \  \ s1.popularidad \ $\leftarrow$ \ s1.popularidad \ + \ s2.popularidad \ + \ 1\\
    2: \  \ s1.antiguedad \ $\leftarrow$ \ max(s1.antiguedad, \ s2.antiguedad)\\
    3: \  \ hijo \ nuevoHijo \ $\leftarrow$ \ $<$direccion(s2), \ s1.turno$>$\\
    4: \  \ agregarAtras(s1.uniones, \ nuevoHijo)\\
    \textbf{Complejidad}: \ O(1)\\
    \makebox[\linewidth]{\rule{\textwidth}{0.4pt}}
    \\
    \makebox[\linewidth]{\rule{\textwidth}{0.4pt}}
    \textbf{mapa}(\textbf{in \ }SimCity \ s) \ $\rightarrow $ \ res \ : \ Mapa \ \\
    1: \  \ Mapa \ res \ $\leftarrow$ \ s.mapa\\
    2: \  \ \textbf{for}(nat \ i \ $\leftarrow$ \ 0; \ i \ $<$ \ long(s.uniones); \ i \ $\leftarrow$ \ i \ + \ 1) \ : \ \\
    3:\indent  \  \ res \ $\leftarrow$ \ res \ + \ \textbf{mapa}(*s.uniones[i].sc)\\
    4: \  \ \textbf{return} \ res\\
    \textbf{Complejidad}: \ O(sumMapas) \ \\
    Cabe \ aclarar \ que \ la \ suma \ de \ los \ mapas \ esta \ definida.\\
    \makebox[\linewidth]{\rule{\textwidth}{0.4pt}}
    \\
    \pagebreak\\
    \makebox[\linewidth]{\rule{\textwidth}{0.4pt}}
    \textbf{listDeTipo}(\textbf{in \ }SimCity \ s, \ \textbf{in \ }construccion \ tipo) \ $\rightarrow $ \ res \ : \ dicc(Pos, \ Nivel)\\
    1: \  \ dicc(pos, \ nivel) \ res \ $\leftarrow$ \ vacio()\\
    2: \  \ \textbf{for}(nat \ i \ $\leftarrow$ \ 0; \ i \ $<$ \ long(s.construcciones); \ i \ $\leftarrow$ \ i \ + \ 1) \ : \ \\
    3:\indent  \  \ itDicc(Pos, \ Construccion) \ itCs \ $\leftarrow$ \ crearIt(*s.construcciones[i]);\\
    4:\indent  \  \ \textbf{while}(haySiguiente(itCs)) \ :\\
    5:\indent \indent  \  \ Pos \ p \ $\leftarrow$ \ siguienteClave(itCs)\\
    6:\indent \indent  \  \ construccion \ c \ $\leftarrow$ \ siguienteSignificado(itCs)\\
    7:\indent \indent  \  \ \textbf{if}(c \ == \ tipo) \ :\\
    8:\indent \indent \indent  \  \ definirRapido(res, \ p, \ s.turno \ - \ (i+1))\\
    9:\indent \indent  \  \ avanzar(itCs) \ \\
    \textbf{Complejidad}: \ O($\sum_{i=1}^{long(s.construcciones)}$($ \# $claves(*s.construcciones[i])))\\
    \textbf{Justificaci\'on:} \ (2) recorre cada uno de los diccionarios definidos en s.construcciones y,\\
    para cada uno, recorre todas las definiciones del mismo: (4). Por lo tanto, la \textbf{complejidad} del\\
    algoritmo es la suma de todas las construcciones definidas en ese SimCity.\\
    \makebox[\linewidth]{\rule{\textwidth}{0.4pt}}
    \\
    \makebox[\linewidth]{\rule{\textwidth}{0.4pt}}
    \textbf{casas}(\textbf{in \ }SimCity \ s) \ $\rightarrow $ \ res \ : \ dicc(Pos, \ Nivel)\\
    1: \  \ dicc(pos, \ nivel) \ res \ $\leftarrow$ \ copiar(listDeTipo(s, \ ``casa''))\\
    2: \  \ dicc(pos, \ nivel) \ comerciosTotales \ $\leftarrow$ \ copiar(listDeTipo(s, \ ``comercio''))\\
    3: \  \ \textbf{for}(nat \ i \ $\leftarrow$ \ 0; \ i \ $<$ \ long(s.uniones); \ i \ $\leftarrow$ \ i \ + \ 1) \ : \ \\
    4:\indent  \  \ itDicc(Pos, \ Nivel) \ itCs \ $\leftarrow$ \ crearIt(\textbf{casas}(*s.uniones[i].sc))\\
    5:\indent  \  \ \textbf{while}(haySiguiente(itCs)) \ :\\
    6:\indent \indent  \  \ Pos \ p \ $\leftarrow$ \ siguienteClave(itCs)\\
    7:\indent \indent  \  \ Nivel \ n \ $\leftarrow$ \ siguienteSignificado(itCs)\\
    8:\indent \indent  \  \ \textbf{if}($\neg$def?(res, \ p) \ $\wedge$ \ $\neg$def?(comerciosTotales, \ p)) \ :\\
    9:\indent \indent \indent  \  \ definirRapido(res, \ p, \ s.turno \ - \ s.uniones[i].turnoUnido \ + \ n)\\
    10:\indent \indent  \ avanzar(itCs)\\
    11:\indent  \ comerciosTotales \ $\leftarrow$ \ comerciosTotales \ $\cup$ \ listDeTipo(*s.uniones[i].sc, \ ``comercio'')\\
    12: \ \textbf{return} \ res\\
    \textbf{Complejidad}: \ O(sumConstrucciones$^2$ \ * \ nodos \ + \ nodos)\\
    \textbf{Justificaci\'on}: \ \\
    Sabemos que este algoritmo va a recorrer, por cada SimCity que pertenece al SimCity inicial, una sola vez. \\
    Ya que el paso recursivo (4) llama a un SimCity, como hijo de otro SimCity, y cada SimCity como \
    máximo es hijo de un solo otro SimCity.\\
    Entonces: nodos\\
    \\
    En cada una de estas recursiones, primero (1) y (2) van a generar un diccionario con las casas \
    y los comercios de ese SimCity particular.\\
    Si repetimos este procedimiento para cada uno de los Simcities que pertenecen al SimCity original, 
    vamos a estar recorriendo una vez las construcciones de cada SimCity. Por lo tanto, la suma de recorrer 
    las construcciones de cada uno es igual a sumConstrucciones.\\
    Entonces: sumConstrucciones\\
    \\
    (3) = $ \# $(hijos de SimCity particular)\\
    (5) = $ \# $(casas totales de un hijo particular)\\
    (8) = $ \# $(comerciosTotales) + $ \# $(casasTotales) = sumConstrucciones\\
    (3) * (5) * (8) = (casasTotales) * sumConstrucciones $\leq$ sumConstrucciones$^2$\\
    Por lo tanto, se hace en el peor caso sumConstrucciones$^2$ para cada SimCity particular. \\
    Entonces: sumConstrucciones$^2$ * nodos\\
    \\
    Luego, en (11), hace la unión de los comercios que ya analizó con los comercios del SimCity que acaba \
    de analizar, pero esta acción la hace una sola vez por SimCity. Pero, como la unión es por copia, y no por \
    referencia, esto es sumConstrucciones$^2$ para cada SimCity particular.\\
    Entonces: sumConstrucciones$^2$ * nodos\\
    \\
    Finalmente: nodos + sumConstrucciones + sumConstrucciones$^2$ * nodos + sumConstrucciones$^2$ * nodos\ $\implies$ sumConstrucciones$^2$ * nodos + nodos\\
    \makebox[\linewidth]{\rule{\textwidth}{0.4pt}} 
    \newpage
    \noindent\makebox[\linewidth]{\rule{\textwidth}{0.4pt}}
    \textbf{comercios}(\textbf{in \ }SimCity \ s) \ $\rightarrow $ \ res \ : \ dicc(Pos, \ Nivel)\\
    1: \  \ dicc(Pos, \ Nivel) \ casasTotales \ $\leftarrow$ \ casas(s)\\
    2: \  \ dicc(Pos, \ Nivel) \ comerciosTotales \ $\leftarrow$ \ comerciosAux(s, \ casasTotales)\\
    2: \  \ \textbf{return} \ manhatizar(comerciosTotales, \ casasTotales)\\
    \textbf{Complejidad}: \ O(manhatizar(comerciosAux(s, \ casasTotales), \ casasTotales) \ + \ casas(s))\\
    \makebox[\linewidth]{\rule{\textwidth}{0.4pt}}
    \\
    \makebox[\linewidth]{\rule{\textwidth}{0.4pt}}
    \textbf{comerciosAux}(\textbf{inout} \ SimCity \ s, \ \textbf{inout} \ casasTotales) \ $\rightarrow $ \ res \ : \ dicc(Pos, \ Nivel)\\
    1: \  \ dicc(pos, \ nivel) \ res \ $\leftarrow$ \ copiar(listDeTipo(s, \ ``comercio''))\\
    3: \  \ \textbf{for}(nat \ i \ $\leftarrow$ \ 0; \ i \ $<$ \ long(s.uniones); \ i \ $\leftarrow$ \ i \ + \ 1) \ : \ \\
    4:\indent  \  \ itDicc(Pos, \ Nivel) \ itCs \ $\leftarrow$ \ crearIt(\textbf{comerciosAux}(*s.uniones[i].sc), \ casasTotales)\\
    5:\indent  \  \ \textbf{while}(haySiguiente(itCs)) \ :\\
    6:\indent \indent  \  \ Pos \ p \ $\leftarrow$ \ siguienteClave(itCs)\\
    7:\indent \indent  \  \ Nivel \ n \ $\leftarrow$ \ siguienteSignificado(itCs)\\
    8:\indent \indent  \  \ \textbf{if}($\neg$def?(res, \ p) \ $\wedge$ \ $\neg$def?(casasTotales, \ p)) \ :\\
    9:\indent \indent \indent  \  \ definirRapido(res, \ p, \ s.turno \ - \ s.uniones[i].turnoUnido \ + \ n)\\
    10:\indent \indent  \ avanzar(itCs)\\
    11: \ \textbf{return} \ res\\
    \textbf{Complejidad}: \ O(sumConstrucciones$^2$ \ * \ nodos \ + \ nodos)\\
    \textbf{Justificaci\'on}: El algoritmo es muy similar al de casas por lo tanto su \
    justificacion es muy similar.\\
    \makebox[\linewidth]{\rule{\textwidth}{0.4pt}}
    \\
    \makebox[\linewidth]{\rule{\textwidth}{0.4pt}}
    \textbf{manhatizar}(\textbf{inout} \ dicc(Pos, \ Nivel) \ comercios, \ \textbf{in \ }dicc(Pos, \ Nivel) \ casasTotales) \ \\
    1: \  \ itDicc(Pos, \ Nivel) \ itCs \ $\leftarrow$ \ crearIt(comercios)\\
    2: \  \ \textbf{while}(haySiguiente(itCs)) \ :\\
    3:\indent  \  \ Pos \ p \ $\leftarrow$ \ siguienteClave(itCs)\\
    4:\indent  \  \ Nivel \ n \ $\leftarrow$ \ siguienteSignificado(itCs)\\
    5:\indent  \  \ definir(comercios, \ p, \ max(n, \ nivelCom(p, \ casasTotales)))\\
    \textbf{Complejidad}: \ O(sumConstrucciones$^2$)\\
    \makebox[\linewidth]{\rule{\textwidth}{0.4pt}}
    \\
    \makebox[\linewidth]{\rule{\textwidth}{0.4pt}}
    \textbf{nivelCom}(\textbf{in \ }Pos \ p, \ \textbf{in \ }dicc(pos, \ Nivel) \ cs) \ $\rightarrow $ \ Nat\\
    1: \  \ nat \ maxLvl \ $\leftarrow$ \ 1\\
    2: \  \ \textbf{for}(int \ i \ = \ -3; \ i \ $\leq$ \ 3; \ ++i) \ :\\
    3:\indent  \  \ \textbf{for}(int \ j \ = \ $\mid$i$\mid$-3; \ j \ $\leq$ \ 3-$\mid$i$\mid$; \ ++j) \ :\\
    4:\indent \indent  \  \ \textbf{if}(p.x \ + \ i \ $\geq$ \ 0 \ $\wedge$ \ p.y \ + \ j \ $\geq$ \ 0) \ :\\
    5:\indent \indent \indent  \  \ Pos \ p2 \ $\leftarrow$ \ $<$p.x+i, \ p.y+j$>$ \ \\
    6:\indent \indent \indent  \  \ \textbf{if}(def?(cs, \ p2)) \ :\\
    7:\indent \indent \indent \indent  \  \ maxLvl \ = \ max(maxLvl, \ obtener(cs, \ p2))\\
    8: \  \ \textbf{return} \ maxLvl\\
    \textbf{Complejidad}: \ O($ \# $claves(cs))\\
    \makebox[\linewidth]{\rule{\textwidth}{0.4pt}}
    \\
    \makebox[\linewidth]{\rule{\textwidth}{0.4pt}}
    \textbf{popularidad}(\textbf{in \ }SimCity \ s) \ $\rightarrow $ \ res \ : \ Nat\\
    1: \  \ \textbf{return} \ s.popularidad\\
    \textbf{Complejidad}: \ O(1)\\
    \makebox[\linewidth]{\rule{\textwidth}{0.4pt}}
    \\
    \makebox[\linewidth]{\rule{\textwidth}{0.4pt}}
    \textbf{turnos}(\textbf{in \ }SimCity \ s) \ $\rightarrow $ \ res \ : \ Nat\\
    1: \  \ \textbf{return} \ s.antiguedad\\
    \textbf{Complejidad}: \ O(1)\\
    \makebox[\linewidth]{\rule{\textwidth}{0.4pt}}
    \\
    \pagebreak \\
    \makebox[\linewidth]{\rule{\textwidth}{0.4pt}}
    $\bullet$ \ $\cup$ \ $\bullet$(\textbf{in \ }dicc($\alpha$, \ $\beta$) \ d1, \ \textbf{in \ }dicc($\alpha$, \ $\beta$) \ d2) \ $\rightarrow $ \ res \ : \ dicc($\alpha$, \ $\beta$)\\
    1: \  \ dicc($\alpha$, \ $\beta$) \ res \ = \ copiar(d1)\\
    2: \  \ itDicc($\alpha$, \ $\beta$) \ itCs \ $\leftarrow$ \ crearIt(d2);\\
    3: \  \ \textbf{while}(haySiguiente(itCs)) \ :\\
    4:\indent  \  \ $\alpha$ \ a \ $\leftarrow$ \ siguienteClave(itCs)\\
    5:\indent  \  \ $\beta$ \ b \ $\leftarrow$ \ siguienteSignificado(itCs)\\
    6:\indent  \  \ \textbf{if}($\neg$def?(res, \ a)) \ :\\
    7:\indent \indent  \  \ definirRapido(res, \ a, \ b)\\
    8:\indent  \  \ avanzar(itCs)\\
    9: \  \ \textbf{return} \ res\\
    \textbf{Complejidad}: \ O(copy(d1) \ + \ $ \# $claves(d2))\\
    \makebox[\linewidth]{\rule{\textwidth}{0.4pt}}
    
\end{Algoritmos}

