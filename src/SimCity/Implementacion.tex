\begin{Algoritmos} \\
    \noindent Interpretamos \ que \ el \ \textbf{inout} \ se \ refiere \ a \ que \ determinado \ parametro \ esta \ siendo \ modificado\\
    y \ se \ toma \ por \ contexto \ si \ pasa \ por \ referencia.\\
    \\
    Sea \ U \ $\equiv$ \ \{u$_1$, \ u$_2$, \ ... \ , \ u$_n$\}\\
    tal \ que \ cada \ uno \ de \ esos \ u$_i$ \ son \ los \ simcities \ que \ componen \ al \ simcity \ original\\
    llamamos \ nodos \ : \ $ \# $U\\
    llamamos \ sumConstrucciones \ : \ $\sum_{i=1}^{nodos}$(copiar(u$_i$.construcciones))\\
    llamamos \ sumMapas \ : \ $\sum_{i=1}^{nodos}$(u$_i$.mapa)\\
    \\
    \makebox[\linewidth]{\rule{\textwidth}{0.4pt}}
    \textbf{iniciar}(\textbf{in \ }m: \ Mapa) \ $\longrightarrow$ \ res \ : \ res\\
    1: \  \ res.turno \ $\leftarrow$ \ 0\\
    2: \  \ res.antiguedad \ $\leftarrow$ \ 0\\
    3: \  \ res.popularidad \ $\leftarrow$ \ 0\\
    4: \  \ res.mapa \ $\leftarrow$ \ m\\
    5: \  \ res.construcciones \ $\leftarrow$ \ vacio()\\
    6: \  \ res.uniones \ $\leftarrow$ \ vacia()\\
    7: \  \ \textbf{return} \ res\\
    \textbf{Complejidad}: \ O(copy(m))\\
    \makebox[\linewidth]{\rule{\textwidth}{0.4pt}}
    \\
    \makebox[\linewidth]{\rule{\textwidth}{0.4pt}}
    \textbf{avanzarTurno}(\textbf{inout} \ SimCity \ s, \ \textbf{in \ }dicc(Pos, \ Construccion) \ cs)\\
    1: \  \ s.turno \ $\leftarrow$ \ s.turno \ + \ 1 \ \\
    2: \  \ s.antiguedad \ $\leftarrow$ \ s.antiguedad \ + \ 1 \ \\
    3: \  \ s.construcciones.agregarAtras(direccion(cs))\\
    \textbf{Complejidad}: \ O(1)\\
    \makebox[\linewidth]{\rule{\textwidth}{0.4pt}}
    \\
    \makebox[\linewidth]{\rule{\textwidth}{0.4pt}}
    \textbf{unir}(\textbf{inout} \ SimCity \ s1, \ \textbf{in \ }Simcity \ s2)\\
    1: \  \ s1.popularidad \ $\leftarrow$ \ s1.popularidad \ + \ s2.popularidad \ + \ 1\\
    2: \  \ s1.antiguedad \ $\leftarrow$ \ max(s1.antiguedad, \ s2.antiguedad)\\
    3: \  \ hijo \ nuevoHijo \ $\leftarrow$ \ $<$direccion(s2), \ s1.turno$>$\\
    4: \  \ agregarAtras(s1.uniones, \ nuevoHijo)\\
    \textbf{Complejidad}: \ O(1)\\
    \makebox[\linewidth]{\rule{\textwidth}{0.4pt}}
    \\
    \makebox[\linewidth]{\rule{\textwidth}{0.4pt}}
    \textbf{mapa}(\textbf{in \ }SimCity \ s) \ $\rightarrow $ \ res \ : \ Mapa \ \\
    1: \  \ Mapa \ res \ $\leftarrow$ \ s.mapa\\
    2: \  \ \textbf{for}(nat \ i \ $\leftarrow$ \ 0; \ i \ $<$ \ long(s.uniones); \ i \ $\leftarrow$ \ i \ + \ 1) \ : \ \\
    3:\indent  \  \ res \ $\leftarrow$ \ res \ + \ \textbf{mapa}(*s.uniones[i].sc)\\
    4: \  \ \textbf{return} \ res\\
    \textbf{Complejidad}: \ O(sumMapas) \ \\
    Cabe \ aclarar \ que \ la \ suma \ de \ los \ mapas \ esta \ definida.\\
    \makebox[\linewidth]{\rule{\textwidth}{0.4pt}}
    \\
    \pagebreak\\
    \makebox[\linewidth]{\rule{\textwidth}{0.4pt}}
    \textbf{listDeTipo}(\textbf{in \ }SimCity \ s, \ \textbf{in \ }construccion \ tipo) \ $\rightarrow $ \ res \ : \ dicc(Pos, \ Nivel)\\
    1: \  \ dicc(pos, \ nivel) \ res \ $\leftarrow$ \ vacio()\\
    2: \  \ \textbf{for}(nat \ i \ $\leftarrow$ \ 0; \ i \ $<$ \ long(s.construcciones); \ i \ $\leftarrow$ \ i \ + \ 1) \ : \ \\
    3:\indent  \  \ itDicc(Pos, \ Construccion) \ itCs \ $\leftarrow$ \ crearIt(*s.construcciones[i]);\\
    4:\indent  \  \ \textbf{while}(haySiguiente(itCs)) \ :\\
    5:\indent \indent  \  \ Pos \ p \ $\leftarrow$ \ siguienteClave(itCs)\\
    6:\indent \indent  \  \ construccion \ c \ $\leftarrow$ \ siguienteSignificado(itCs)\\
    7:\indent \indent  \  \ \textbf{if}(c \ == \ tipo) \ :\\
    8:\indent \indent \indent  \  \ definirRapido(res, \ p, \ s.turno \ - \ (i+1))\\
    9:\indent \indent  \  \ avanzar(itCs) \ \\
    \textbf{Complejidad}: \ O($\sum_{i=1}^{long(s.construcciones)}$($ \# $claves(*s.construcciones[i])))\\
    Justificaci\'on: \ (2) \ recorre \ cada \ uno \ de \ los \ diccionarios \ definidos \ en \ s.construcciones \ y\\
    para \ cada \ uno \ recorre \ todas \ las \ definiciones \ del \ mismo, \ (4). \ Por \ lo \ tanto \ la \ \textbf{complejidad} \ del\\
    algoritmo \ es \ la \ suma \ de \ todas \ las \ construcciones \ definidas \ en \ ese \ SimCity.\\
    \makebox[\linewidth]{\rule{\textwidth}{0.4pt}}
    \\
    \makebox[\linewidth]{\rule{\textwidth}{0.4pt}}
    \textbf{casas}(\textbf{in \ }SimCity \ s) \ $\rightarrow $ \ res \ : \ dicc(Pos, \ Nivel)\\
    1: \  \ dicc(pos, \ nivel) \ res \ $\leftarrow$ \ copiar(listDeTipo(s, \ ``casa''))\\
    2: \  \ dicc(pos, \ nivel) \ comerciosTotales \ $\leftarrow$ \ copiar(listDeTipo(s, \ ``comercio''))\\
    3: \  \ \textbf{for}(nat \ i \ $\leftarrow$ \ 0; \ i \ $<$ \ long(s.uniones); \ i \ $\leftarrow$ \ i \ + \ 1) \ : \ \\
    4:\indent  \  \ itDicc(Pos, \ Nivel) \ itCs \ $\leftarrow$ \ crearIt(\textbf{casas}(*s.uniones[i].sc))\\
    5:\indent  \  \ \textbf{while}(haySiguiente(itCs)) \ :\\
    6:\indent \indent  \  \ Pos \ p \ $\leftarrow$ \ siguienteClave(itCs)\\
    7:\indent \indent  \  \ Nivel \ n \ $\leftarrow$ \ siguienteSignificado(itCs)\\
    8:\indent \indent  \  \ \textbf{if}($\neg$def?(res, \ p) \ $\wedge$ \ $\neg$def?(comerciosTotales, \ p)) \ :\\
    9:\indent \indent \indent  \  \ definirRapido(res, \ p, \ s.turno \ - \ s.uniones[i].turnoUnido \ + \ n)\\
    10:\indent \indent  \ avanzar(itCs)\\
    11:\indent  \ comerciosTotales \ $\leftarrow$ \ comerciosTotales \ $\cup$ \ listDeTipo(*s.uniones[i].sc, \ ``comercio'')\\
    12: \ \textbf{return} \ res\\
    \textbf{Complejidad}: \ O(sumConstrucciones$^2$ \ * \ nodos \ + \ nodos)\\
    \textbf{Justificaci\'on}: \ \\
    Sabemos \ que \ este \ algoritmo \ va \ a \ recorrer \ por \ cada \ SimCity \ que \ pertenece \ al \ SimCity \ inicial \ 1 \ vez. \ \\
    Ya \ que \ el \ paso \ recursivo \ (4) \ llama \ a \ un \ simcity \ como \ hijo \ de \ otro \ simcity \ y \ cada \ simcity \ como \
    maximo \ es \ hijo \ de \ 1 \ solo \ otro \ simcity\\
    --$>$ \ nodos\\
    \\
    En \ cada \ una \ de \ estas \ recursiones, \ primero \ (1) \ y \ (2) \ van \ a \ generar \ un \ diccionario \ con \ las \ casas \
    y \ los \ comercios \ de \ ese \ simcity \ particular.\\
    Si \ repetimos \ este \ procedimiento \ para \ cada \ uno \ de \ los \ Simcities \ que \ pertenecen \ al \ simcity \ original \ 
    vamos \ a \ estar \ recorriendo \ 1 \ vez \ las \ construcciones \ de \ cada \ simcity \ por \ lo \ tanto \ la \ suma \ de \ recorrer \ 
    las \ construcciones \ de \ cada \ uno \ es \ igual \ a \ sumConstrucciones.\\
    --$>$ \ sumConstrucciones\\
    \\
    (3) \ = \ $ \# $(hijos \ de \ simCity \ particular)\\
    (5) \ = \ $ \# $(casas \ totales \ de \ un \ hijo \ particular)\\
    (8) \ = \ $ \# $(comerciosTotales) \ + \ $ \# $(casasTotales) \ = \ sumConstrucciones\\
    (3) \ * \ (5) \ * \ (8) \ = \ (casasTotales) \ * \ sumConstrucciones \ $\leq$ \ sumConstrucciones$^2$\\
    por \ lo \ tanto \ si \ hace \ en \ el \ peor \ caso \ sumConstrucciones$^2$ \ para \ cada \ Simcity \ particular \ \\
    --$>$ \ sumConstrucciones$^2$ \ * \ nodos\\
    \\
    luego \ en \ (11) \ hace \ la \ union \ de \ los \ comercios \ que \ ya \ analizo \ con \ los \ comercios \ del \ simcity \ que \ acaba \
    de \ analizar \ pero \ esta \ accion \ la \ hace \ 1 \ sola \ vez \ por \ simcity. \ Pero \ como \ la \ union \ es \ por \ copia \ y \ no \ por \
    referencia \ esto \ es \ sumConstrucciones$^2$ \ para \ cada \ simcity \ particular.\\
    --$>$ \ sumConstrucciones$^2$ \ * \ nodos\\
    \\
    finalmente: \ nodos \ + \ sumConstrucciones \ + \ sumConstrucciones$^2$ \ * \ nodos \ + \ sumConstrucciones$^2$ \ * \ nodos\\
    --$>$ \ sumConstrucciones$^2$ \ * \ nodos \ + \ nodos\\
    \makebox[\linewidth]{\rule{\textwidth}{0.4pt}} 

    \noindent\makebox[\linewidth]{\rule{\textwidth}{0.4pt}}
    \textbf{comercios}(\textbf{in \ }SimCity \ s) \ $\rightarrow $ \ res \ : \ dicc(Pos, \ Nivel)\\
    1: \  \ dicc(Pos, \ Nivel) \ casasTotales \ $\leftarrow$ \ casas(s)\\
    2: \  \ dicc(Pos, \ Nivel) \ comerciosTotales \ $\leftarrow$ \ comerciosAux(s, \ casasTotales)\\
    2: \  \ \textbf{return} \ manhatizar(comerciosTotales, \ casasTotales)\\
    \textbf{Complejidad}: \ O(manhatizar(comerciosAux(s, \ casasTotales), \ casasTotales) \ + \ casas(s))\\
    \\
    \textbf{comerciosAux}(\textbf{inout} \ SimCity \ s, \ \textbf{inout} \ casasTotales) \ $\rightarrow $ \ res \ : \ dicc(Pos, \ Nivel)\\
    1: \  \ dicc(pos, \ nivel) \ res \ $\leftarrow$ \ copiar(listDeTipo(s, \ ``comercio''))\\
    3: \  \ \textbf{for}(nat \ i \ $\leftarrow$ \ 0; \ i \ $<$ \ long(s.uniones); \ i \ $\leftarrow$ \ i \ + \ 1) \ : \ \\
    4:\indent  \  \ itDicc(Pos, \ Nivel) \ itCs \ $\leftarrow$ \ crearIt(\textbf{comerciosAux}(*s.uniones[i].sc), \ casasTotales)\\
    5:\indent  \  \ \textbf{while}(haySiguiente(itCs)) \ :\\
    6:\indent \indent  \  \ Pos \ p \ $\leftarrow$ \ siguienteClave(itCs)\\
    7:\indent \indent  \  \ Nivel \ n \ $\leftarrow$ \ siguienteSignificado(itCs)\\
    8:\indent \indent  \  \ \textbf{if}($\neg$def?(res, \ p) \ $\wedge$ \ $\neg$def?(casasTotales, \ p)) \ :\\
    9:\indent \indent \indent  \  \ definirRapido(res, \ p, \ s.turno \ - \ s.uniones[i].turnoUnido \ + \ n)\\
    10:\indent \indent  \ avanzar(itCs)\\
    11: \ \textbf{return} \ res\\
    \textbf{Complejidad}: \ O(sumConstrucciones$^2$ \ * \ nodos \ + \ nodos)\\
    \textbf{Justificaci\'on}: \ El \ algoritmo \ es \ muy \ similar \ al \ de \ casas \ por \ lo \ tanto \ su \
    justificacion \ es \ muy \ similar\\
    \\
    \textbf{manhatizar}(\textbf{inout} \ dicc(Pos, \ Nivel) \ comercios, \ \textbf{in \ }dicc(Pos, \ Nivel) \ casasTotales) \ \\
    1: \  \ itDicc(Pos, \ Nivel) \ itCs \ $\leftarrow$ \ crearIt(comercios)\\
    2: \  \ \textbf{while}(haySiguiente(itCs)) \ :\\
    3:\indent  \  \ Pos \ p \ $\leftarrow$ \ siguienteClave(itCs)\\
    4:\indent  \  \ Nivel \ n \ $\leftarrow$ \ siguienteSignificado(itCs)\\
    5:\indent  \  \ definir(comercios, \ p, \ max(n, \ nivelCom(p, \ casasTotales)))\\
    \textbf{Complejidad}: \ O(sumConstrucciones$^2$)\\
    \makebox[\linewidth]{\rule{\textwidth}{0.4pt}}
    \\
    \makebox[\linewidth]{\rule{\textwidth}{0.4pt}}
    \textbf{nivelCom}(\textbf{in \ }Pos \ p, \ \textbf{in \ }dicc(pos, \ Nivel) \ cs) \ $\rightarrow $ \ Nat\\
    1: \  \ nat \ maxLvl \ $\leftarrow$ \ 1\\
    2: \  \ \textbf{for}(int \ i \ = \ -3; \ i \ $\leq$ \ 3; \ ++i) \ :\\
    3:\indent  \  \ \textbf{for}(int \ j \ = \ $\mid$i$\mid$-3; \ j \ $\leq$ \ 3-$\mid$i$\mid$; \ ++j) \ :\\
    4:\indent \indent  \  \ \textbf{if}(p.x \ + \ i \ $\geq$ \ 0 \ $\wedge$ \ p.y \ + \ j \ $\geq$ \ 0) \ :\\
    5:\indent \indent \indent  \  \ Pos \ p2 \ $\leftarrow$ \ $<$p.x+i, \ p.y+j$>$ \ \\
    6:\indent \indent \indent  \  \ \textbf{if}(def?(cs, \ p2)) \ :\\
    7:\indent \indent \indent \indent  \  \ maxLvl \ = \ max(maxLvl, \ obtener(cs, \ p2))\\
    8: \  \ \textbf{return} \ maxLvl\\
    \textbf{Complejidad}: \ O($ \# $claves(cs))\\
    \makebox[\linewidth]{\rule{\textwidth}{0.4pt}}
    \\
    \makebox[\linewidth]{\rule{\textwidth}{0.4pt}}
    \textbf{popularidad}(\textbf{in \ }SimCity \ s) \ $\rightarrow $ \ res \ : \ Nat\\
    1: \  \ \textbf{return} \ s.popularidad\\
    \textbf{Complejidad}: \ O(1)\\
    \makebox[\linewidth]{\rule{\textwidth}{0.4pt}}
    \\
    \makebox[\linewidth]{\rule{\textwidth}{0.4pt}}
    \textbf{turnos}(\textbf{in \ }SimCity \ s) \ $\rightarrow $ \ res \ : \ Nat\\
    1: \  \ \textbf{return} \ s.antiguedad\\
    \textbf{Complejidad}: \ O(1)\\
    \makebox[\linewidth]{\rule{\textwidth}{0.4pt}}
    \\
    \pagebreak \\
    \makebox[\linewidth]{\rule{\textwidth}{0.4pt}}
    $\bullet$ \ $\cup$ \ $\bullet$(\textbf{in \ }dicc($\alpha$, \ $\beta$) \ d1, \ \textbf{in \ }dicc($\alpha$, \ $\beta$) \ d2) \ $\rightarrow $ \ res \ : \ dicc($\alpha$, \ $\beta$)\\
    1: \  \ dicc($\alpha$, \ $\beta$) \ res \ = \ copiar(d1)\\
    2: \  \ itDicc($\alpha$, \ $\beta$) \ itCs \ $\leftarrow$ \ crearIt(d2);\\
    3: \  \ \textbf{while}(haySiguiente(itCs)) \ :\\
    4:\indent  \  \ $\alpha$ \ a \ $\leftarrow$ \ siguienteClave(itCs)\\
    5:\indent  \  \ $\beta$ \ b \ $\leftarrow$ \ siguienteSignificado(itCs)\\
    6:\indent  \  \ \textbf{if}($\neg$def?(res, \ a)) \ :\\
    7:\indent \indent  \  \ definirRapido(res, \ a, \ b)\\
    8:\indent  \  \ avanzar(itCs)\\
    9: \  \ \textbf{return} \ res\\
    \textbf{Complejidad}: \ O(copy(d1) \ + \ $ \# $claves(d2))\\
    \makebox[\linewidth]{\rule{\textwidth}{0.4pt}}
    
\end{Algoritmos}

