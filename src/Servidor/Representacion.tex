\begin{Representacion}
  
    \Titulo{Representación de servidor}
  
    Un servidor almacena y actualiza los diferentes SimCity. Se representa como un 
    diccionario implementado en un trie, donde las claves son los nombres de las partidas
    y los significados un puntero al SimCity y su estado (si es modificable o no).\\

    Elegimos esta estructura para cumplir con las restricciones dadas de complejidad. Un diccionario implementado en un trie cuenta con
    las operaciones propias de un diccionario (definir, definido?, borrar, significado, iteradores), pero con complejidades acotadas por la 
    clave más larga definida en el mismo. En particular, buscar una clave y obtener su significado en este diccionario implica recorrer
    el trie, lo cual en el peor caso implica realizar $|nombreMasLargo|$ comparaciones. Con lo cual, todas las operaciones del servidor 
    en relación a una partida específica serán por lo menos $\bold{O}(|nombreMasLargo|)$.

    

    \begin{Estructura}{servidor}[diccTrie(Nombre, partida)]
        \begin{Tupla}[partida]
            \tupItem{modificable\ }{bool}
            \tupItem{\\sim\ }{puntero(SimCity)}
        \end{Tupla}

    \end{Estructura}
    
    \vspace{2mm}
    \Rep[estr]{ \\
            ($\forall\ partida_{1},\ partida_{2}$\ : Nombre) \\
                ((def?($partida_{1},\ e$) $\wedge$ def?($partida_{2},\ e$) $\wedge\ partida_{1} \neq partida_{2}$) \impluego \\
                \tab obtener($partida_{1},\ e$).sim $\neq$ obtener($partida_{2},\ e$).sim \\
            )\ $\wedge$ \\
            ($\forall\ partida$: Nombre)(def?($partida,\ e$) \impluego\ obtener($partida,\ e$).sim $\neq$ NULL)
      }
    \vspace{2mm}

    ~
    
    \AbsFc[estr]{servidor}[e]{
        s: servidor | \\
            \tab ($\forall\ nombre$: Nombre)\\
                \tab\tab ($nombre$ $\in$ congelados($s$) $\Leftrightarrow$ \\
                \tab\tab (def?($nombre,\ e$) \yluego\ $\neg$ obtener($nombre,\ e$).modificable)) \\
            \tab $\wedge$ \\
            \tab ($\forall\ nombre$: Nombre)\\
                \tab\tab (def?($nombre$, partidas($s$)) $\Leftrightarrow$ def?($nombre,\ e$)) \\
            \tab \yluego \\
            \tab ($\forall\ nombre$: Nombre)\\
                \tab\tab (def?($nombre$, partidas($s$)) \impluego \\ 
                \tab\tab obtener($nombre$, partidas($s$)) $\igobs$ *(obtener($nombre,\ e$).sim)) \\
    }  

  \end{Representacion}
  
